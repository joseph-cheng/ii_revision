\documentclass[12pt]{article}
\usepackage[margin=1.5cm]{geometry}
\usepackage{parskip}
\usepackage{amsmath}
\usepackage{amssymb}
\usepackage{amsfonts}
\usepackage{enumitem}
\usepackage{graphicx}
\usepackage{stmaryrd}
\graphicspath{ {./images/} }


\begin{document}
\begin{enumerate}[label=(\alph*)]

  \item

    \begin{enumerate}[label=(\roman*)]

      \item

        Phylogeny algorithms aim to compute phylogenetic trees, which represent relationships/closeness of different species, from DNA sequences of species (or some proxy metric for a DNA sequence).

        The two types of algorithms we cover are parsimony-based algorithms and distance-based algorithms.

        There are two problems within parsimony-based algorithms: small parsimony and large parsimony.

        Within small parsimony, we take as input the DNA sequences of the trees, as well as a rooted binary tree with labels at each of the leaves, for each of the species. The algorithm then returns a labelling of the intermediate nodes with DNA sequences such that the labelling minimises the overall parsimony, where the overall parsimony is the sum of the state changes we have to make to go from a child node to a parent node (i.e. the number of bases in the DNA sequence we have to change). The large parsimony problem is the same, but has to find the tree and initial labelling itself.

        Distance-based phylogeny algorithms take as input a distance matrix between the different species, computed for example from similarity scores of the DNA sequences. Using this distance matrix, a tree is constructed, and edge lengths are given.

        We can see that distance-based algorithms take a different input to parsimony-based algorithms, and always construct the structure of the tree themselves, but potentially lose information due to using a distance matrix instead of the raw DNA sequences themselves (although for the gain of performance).

        \item
          A distance-based phylogeny algorithm takes as input a distance matrix between all of the species. This could be computed, for example, from an inversion of the similarity scores created during pairwise alignment of each of the species' DNA sequences. 

          As output, the algorithm returns either a rooted or unrooted tree, where each of the leaves are labelled with a particular species, and edges are given lengths representing some kind of distance to neighbouring nodes. In the case of algorithms like UPGMA, which generate ultrametric trees, these distances can be interpreted as time under the assumption of a molecular clock. If the distance matrix is additive, some algorithms (like additive phylogeny and neighbour-joining) will produce the simple tree that fits the matrix, but algorithms like UPGMA will not necessarily generate such a tree.

          \item
            The neighbour joining algorithm works in iterations.

            At a particular iteration, suppose we have $k$ leaves/nodes left in our diagram. At this point, from our distance matrix, we compute an adjusted distance matrix. To compute a new value for a cell, we use the row sums, which take $O(k^2)$ time to compute, and then once we have this information, computing the new adjusted distance value takes $O(1)$ for each node, so computing this matrix takes $O(k^2)$ time.

            Then, we find the minimum element, remove it ($O(1)$ time), and create a new distance matrix with the two joined nodes replaced by a new node, creating a $k-1 \times k-1$ distance matrix, and taking time $O(k)$ to compute, so an entire iteration takes $O(k^2)$ time.

          Since an iteration removes a single node and we terminate when we have $2$ nodes, we have $n-3$ iterations, so if we have $n$ species to begin with, our time complexity is $O(\sum_{k=0}^{n-2} k^2) = O(n^3)$.



        
    \end{enumerate}

    \item
The Needleman-Wunsch algorithm is a sequence alignment algorithm.

The sequence alignment problem is the problem of finding the best alignment between two sequences such that regions of similarity are conserved. For example, this can help to find similarities between DNA sequences of two different species.

In particular, the Needleman-Wunsch algorithm solves the global alignment problem, which is where we align two entire strings against each other.

The algorithm is a dynamic programming algorithm, so if we are trying to align two sequences $x$ and $y$, if we have alignments of prefixes of $x$ and $y$, we can align these prefixes plus 1 character, for example.

The algorithm takes as input two strings $x, y$, a gap penalty $\delta$, and a scoring matrix $s$ (which gives the match/mismatch score between two characters), and returns as output an alignment score and alignment.

Then, we compute the following function/matrix with dynamic programming:

\[
  F(i,j) = \max\begin{cases}F(i-1, j) - \delta\\F(i, j-1) - \delta\\F(i-1, j-1)+ s(x_i, y_j)\end{cases}
.\] 

With initialisation of $F(i,0) = -i \cdot \delta$ and $F(0, j) = -j \cdot \delta$ for all $i$ and $j$.

We also store an additional $Ptr$ matrix that tells us in what direction to go to trace back an alignment.

Then, we get the alignment score in $F(|x|, |y|)$, and backtrack using the $Ptr$ matrix to obtain an alignment.

Consider the following example:

Suppose we are aligning DNA sequences $ACCAGT$ and $CCAGGT$, with $\delta = 1$ and $s(x_i, y_j) = 1$ if $x_i = y_j$, and $-2$ otherwise.

We build the following matrix:

\begin{tabular}{c|c|c|c|c|c|c|c}
  &$\omega$&A&C&C&A&G&T\\
  \hline
  $\omega$&0&-1&-2&-3&-4&-5&-6\\
  \hline
  C&-1&-2&0&-1&-2&-3&-4\\
  \hline
  C&-2&-3&-1&1&0&-1&-2\\
  \hline
  A&-3&-1&-2&0&2&1&0\\
  \hline
  G&-4&-2&-3&-1&1&3&2\\
  \hline
  G&-5&-3&-4&-2&0&2&1\\
  \hline
  T&-6&-4&-5&-3&-1&1&3
\end{tabular}

By backtracking from the bottom-right corner, we get the alignment:

\begin{verbatim}
ACCA-GT
-CCAGGT
\end{verbatim}

Note that this is a unique alignment, but sometimes there will be multiple alignments with the same alignment score.

A naive implementation of this algorithm has $O(nm)$ time complexity and $O(nm)$ space complexity to fill the matrix (where $n = |x|$ and $m = |y|$).
        
    \end{enumerate}
\end{document}

