\documentclass[12pt]{article}
\usepackage[margin=1.5cm]{geometry}
\usepackage{parskip}
\usepackage{amsmath}
\usepackage{amssymb}
\usepackage{amsfonts}
\usepackage{enumitem}
\usepackage{graphicx}
\usepackage{stmaryrd}
\graphicspath{ {./images/} }


\begin{document}
\begin{enumerate}[label=(\alph*)]
  \item Not relevant

  \item
    In Clustal, we use a progressive alignment to generate a multiple alignment. If we do not use a guide tree, then we may pick any arbitrary order to proceed with the multiple alignment.

    Suppose we have sequences \texttt{CCGT}, \texttt{TAAT}, and \texttt{TAAG}

    If we do these in a bad order, we might obtain the following multi-alignment:

\begin{verbatim}
---TAAT---
CCG---T---
------TAAG
\end{verbatim}

However, by doing them in a better order (e.g. by joining the similar sequences first), we might get:

\begin{verbatim}
---TAAT
---TAAG
CCGT---
\end{verbatim}

A guide tree allows us to try and get a better order, so fewer errors are propagated down the multiple alignment.

\item
  Not relevant.

\item
  A compression algorithm is often used in genome assembly because the sheer quantity of data needed to store is huge, and so the memory requirements of these algorithms become huge, so a compression algorithm is used to lessen these requirements.

  For example, the Burrows-Wheeler Transform (BWT) is a reversible transform that tries to amplify similarities in the text, e.g. by placing the same character next to each other often. The BWT makes a string amenable to further compression with something like RLE, or adaptive types of compression, like adaptive arithmetic coding. Importantly, the BWT is also reversible.

  As an example, we might transform the string \texttt{ACAGACAT\$} into \texttt{T\$GCCAAAA}. Clearly, the latter is much more amenable to RLE.

        
\end{enumerate}
\end{document}
