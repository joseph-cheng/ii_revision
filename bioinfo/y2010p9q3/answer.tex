\documentclass[12pt]{article}
\usepackage[margin=1.5cm]{geometry}
\usepackage{parskip}
\usepackage{amsmath}
\usepackage{amssymb}
\usepackage{amsfonts}
\usepackage{enumitem}
\usepackage{graphicx}
\usepackage{stmaryrd}
\graphicspath{ {./images/} }


\begin{document}
\begin{enumerate}[label=(\alph*)]
  \item
    Not relevant.

  \item
    Sankoff's parsimony method is a technique to find the most parsimonious labelling of the internal nodes of an unrooted tree.

    The algorithm is a dynamic programming algorithm, and works character-by-character. It is a dynamic programming algorithm. At a particular node in a tree, we calculate a $k$-length array (where $k$ is the number of different characters we can have), representing the parsimony score for that node if it is assigned a particular character. To calculate one entry in this array, we must find the minimum cost from looking at the parsimony arrays of the node's two children, which takes $O(k)$ time. We must fill in a $k$-length array, so filling in the array takes $O(k^2)$ time. We must do this once per internal node, and if we have $n$ leaves, we get $n-1$ internal nodes, so doing this for one character position takes $O(nk^2)$ time. Finally, for the whole algorithm, we have to do this one per character position, which takes $O(mnk^2)$ is $m$ is the number of character positions (length of each sequence).

  \item
    Not relevant

  \item
    The Doob-Gillespie algorithm is an algorithm used to simulate reactions.
    
    There are a few assumptions:

    \begin{itemize}
      \item
        We assume that the molecules are well-mixed, i.e. are uniformly distributed throughout the mixture, such that probability of reaction is proportional to the number of each molecule (and propensity).

      \item
        We assume that reactions occur only serially, i.e. not in parallel.
    \end{itemize}

    Using these two assumptions, we can develop the algorithm:

    Create a set of possible reactions.

    Calculate propensity values for each reaction, which are proportional to the probabilities that they occur.

    Sum together the propensities to create $a_0$. The higher this number is, the sooner the next reaction will occur.

    Choose the time the next reaction will occur $\tau$ by sampling an exponential random variable with parameter $a_0$.

    By weighting each propensity function by $a_0$, use them as probabilities in a discrete probability distribution, and sample a reaction.

    Update the time by $\tau$, and calculate the new molecule numbers according to the reaction chosen and $\tau$.

    Repeat.




    
\end{enumerate}
\end{document}
