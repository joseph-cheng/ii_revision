\documentclass[12pt]{article}
\usepackage[margin=1.5cm]{geometry}
\usepackage{parskip}
\usepackage{amsmath}
\usepackage{amssymb}
\usepackage{amsfonts}
\usepackage{enumitem}
\usepackage{graphicx}
\usepackage{stmaryrd}
\graphicspath{ {./images/} }


\begin{document}
\begin{enumerate}[label=(\alph*)]
  \item
    \begin{enumerate}[label=(\roman*)]
      \item

        We use dynamics programming algorithms for sequence alignment because sequence alignment problems can be reduced to solving a set of overlapping sub-problems.

        In the case of pairwise sequence alignment, if we have sequences $x$ and $y$, and we want to align them starting at the $i$th and $j$th index respectively, then the best alignment score $F(i,j)$ comes from:

        $F(i,j) = \max \begin{cases}F(i, j-1) - \delta\\F(i-1, j) - \delta\\F(i-1, j-1) + \epsilon & x_i=y_j\\F(i-1, j-1) - \epsilon' & x_i\neq y_j\end{cases}$

        Where $\delta$ is the penalty for inserting a gap, $\epsilon$ is the reward for a match, and $\epsilon'$ is the penalty for a mismatch.

        The use of dynamic programming in computing $F$ means that this only takes $O(|x| \cdot |y|)$ time.

      \item
        In DNA sequence alignment, we might want to reduce the penalty of gaps if we are in a long chain, since we expect that when gaps occur, they occur in large contiguous bunches, so we do not necessarily want to penalise large gaps significantly more than smaller gaps.

        To do this, we might use a convex gap penalty function instead of a linear one, but this increases the asymptotic runtime, so we might instead use affine gaps, where we have a large gap opening penalty, and a smaller gap extension penalty.

        
    \end{enumerate}

  \item
    \begin{enumerate}[label=(\roman*)]
      \item
        (PatternHunter and BLAST not relevant)

        The Smith-Waterman and Needleman-Wunsch algorithms are both algorithms for sequence alignment, but the Smith-Waterman algorithm is an algorithm for local alignment, i.e. finding closely-matching subsequences between two sequences, and the Needleman-Wunsch algorithm is for global alignment, i.e. finding an alignment between two entire sequences.

        The Needleman-Wunsch algorithm works similarly to described in (a)(i): we decide on an indel penalty $d$, and a score matrix $s$, and solve with dynamic programming the following:

        $F(i,j) = \max \begin{cases}F(i-1, j) - \delta\\F(i, j-1) - \delta\\F(i-1, j-1) + s(x_i, y_j)\end{cases}$

        We do this by building a $|x| \times |y|$ matrix, where the bottom-right cell is the final alignment score.

        Alongside, we build a pointer matrix that essentially stores the direction we moved in to get to that cell, allowing us to reconstruct an alignment.

        The Smith-Waterman algorithm is similar, but we instead add an additional case into our $F$:

        $F(i,j) = \max \begin{cases}0\\F(i-1, j) - \delta\\F(i, j-1) - \delta\\F(i-1, j-1) + s(x_i, y_j)\end{cases}$

        This extra case ensures that we are always looking at the best local case, and not being punished by arriving at a particular $(i,j)$.
    \end{enumerate}

  \item
    Dynamic programming can be easily extended to multiple alignment, but at a cost: runtime complexity. For an $n$-alignment, it runs in time $O(|x|^n)$, where $|x|$ is the maximum length of the strings in the set, so we essentially have exponential run-time.

    Other techniques, like CLUSTAL, use heuristics to speed up the alignment, but at the cost of not necessarily being optimal.

    The idea behind CLUSTAL is the idea that a multiple alignment induces pairwise alignments, and so by constructing each of the pairwise alignments (which is reasonably fast), we can construct a tree that groups similar sequences together (by turning the alignment scores into distance measures, attempting to represent evolutionary distance, and then e.g. using Neighbour-Joining), and then using this tree to guide an order of iterative multiple alignment: where we create a multiple alignment by aligning each sequence in turn to our overall multiple alignment so far.

    This has runtime $O(|x|^2n^2)$, which is much faster than the dynamic programming method.

        
\end{enumerate}
\end{document}
