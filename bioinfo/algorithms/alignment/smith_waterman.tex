\documentclass[12pt]{article}
\usepackage[margin=1.5cm]{geometry}
\usepackage{parskip}
\usepackage{amsmath}
\usepackage{amssymb}
\usepackage{amsfonts}
\usepackage{enumitem}
\usepackage{graphicx}
\usepackage{stmaryrd}
\graphicspath{ {./images/} }


\begin{document}
The Smith-Waterman algorithm is a sequence alignment algorithm.

The sequence alignment problem is the problem of finding the best alignment between two sequences such that regions of similarity are conserved. For example, this can help to find similarities between DNA sequences of two different species.

In particular, the Smith-Waterman algorithm solves the local alignment problem, which is where we align in subrectangles of the edit matrix.

The algorithm is a dynamic programming algorithm, so if we are trying to align two sequences $x$ and $y$, if we have alignments of prefixes of $x$ and $y$, we can align these prefixes plus 1 character, for example.

The algorithm takes as input two strings $x, y$, a gap penalty $\delta$, and a scoring matrix $s$ (which gives the match/mismatch score between two characters), and returns as output an alignment score and alignment.

Then, we compute the following function/matrix with dynamic programming:

\[
  F(i,j) = \max\begin{cases}0\\F(i-1, j) - \delta\\F(i, j-1) - \delta\\F(i-1, j-1)+ s(x_i, y_j)\end{cases}
.\] 

With initialisation of $F(i,0) = 0$ and $F(0, j) = 0$ for all $i$ and $j$.

We also store an additional $Ptr$ matrix that tells us in what direction to go to trace back an alignment.

The addition of 0 into the cases of the dynamic programming function is what makes this algorithm compute local alignment: at any point we are allowed to reset to consider a new subrectangle.

Then, we get the alignment score in $F(|x|, |y|)$, and backtrack using the $Ptr$ matrix to obtain an alignment.

Consider the following example:

Suppose we are aligning DNA sequences $ACCAGT$ and $CCAGGT$, with $\delta = 1$ and $s(x_i, y_j) = 1$ if $x_i = y_j$, and $-2$ otherwise.

We build the following matrix:

\begin{tabular}{c|c|c|c|c|c|c|c}
  &$\omega$&A&C&C&A&G&T\\
  \hline
  $\omega$&0&0&0&0&0&0&0\\
  \hline
  C&0&0&1&1&0&0&0\\
  \hline
  C&0&0&1&2&1&0&0\\
  \hline
  A&0&1&0&1&3&2&1\\
  \hline
  G&0&0&0&0&2&4&3\\
  \hline
  G&0&0&0&0&1&3&2\\
  \hline
  T&0&0&0&0&0&2&4
\end{tabular}


This can be used, for example, to identify common subsequences \texttt{CCAG} in both sequences.

This algorithm has $O(nm)$ time complexity and $O(nm)$ space complexity to fill the matrix (where $n = |x|$ and $m = |y|$).

The algorithm makes the potentially unreasonable assumption that gaps should be penalised linearly, when in reality sequences of gaps likely occur in a single biological event, not a combination of them. So, we would like to use a concave gap penalty function, but this increases the time complexity to $O(n^3)$, so we make the tradeoff of using an affine gap penalty, which has a high gap opening penalty, but a small gap extension penalty.






\end{document}
