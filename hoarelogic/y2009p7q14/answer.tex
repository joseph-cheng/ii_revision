\documentclass[12pt]{article}
\usepackage[margin=1.5cm]{geometry}
\usepackage{parskip}
\usepackage{amsmath}
\usepackage{amssymb}
\usepackage{amsfonts}
\usepackage{enumitem}
\usepackage{graphicx}
\usepackage{stmaryrd}
\graphicspath{ {./images/} }


\begin{document}
\begin{enumerate}[label=(\alph*)]

  \item
    A deductive system makes truth judgements $\vdash P$ about assertions $P$. This will correspond to truth judgments $\vDash P$ in the real-world that the deductive system is modelling.

    A deductive system is sound if whenever $\vdash P$, we have $\vDash P$. In essence, this means that our deductive system will not make incorrect claims about the truth of an assertion.

    A deductive system is complete if whenever $\vDash P$, we can show $\vdash P$. In essence, this means that our deductive system can prove everything that is true in the real-world.

    \item
      Soundness is thought of as more important than completeness because if our logic is not sound, then proving $\vdash P$ does not really allow us to do anything useful, since we do not know whether or not $\vDash P$ is true, and ultimately we care about making decisions based on truths in the real-world, as opposed to our deductive system (which is really just a means to say useful things about the real world).

      Completeness is useful, since it allows us to prove everything we could want, but without completeness we can still make claims about the real world using our deductive system.

      \item
\begin{verbatim}
{X = x}
if (arith(X) = 0) THEN X := 1 ELSE SKIP
if (arith2(X) != 0) THEN X := 1 ELSE SKIP
{X = 1}
\end{verbatim}

The above triple is unprovable, assuming \texttt{arith} and \texttt{arith2} are some arbitrary arithmetic functions of \texttt{X} that are equivalent, but not syntactically, due to Hoare logic only being relatively complete. Since arithmetic is undecidable in general, we cannot determine that \texttt{arith(X)} and \texttt{arith2(X)} are equivalent, and thus we cannot determine that we will always end with $X=1$, unless we assume an oracle for our assertions.

\item
  We propose:

  $C = X := 1$

  $P = X = X$

  $Q = X = 1$

  $R = X = 0$

  Clearly, we have $\{X=X\}X := 1\{X = 1\}$ by application of the assignment rule.

  If we had this rule, then we could derive $\{X=X \wedge X=0\}X := 1\{X = 1 \wedge X = 0\}$.

  This triple is clearly not true, since it terminates, and the postcondition has to be false, and therefore this rule cannot be sound because we have $\vdash \{P \wedge R\}C\{Q \wedge R\}$ but not $\vDash \{P \wedge R \}C\{Q \wedge R\}$.

  \item
    We have soundness as long as $C$ does not affect any variables that $R$ makes claims about, i.e. $fv(C) \cap fv(R) = \{\}$.

    These conditions ensure soundness if $C$ cannot change any of the state that $R$ makes claims about, then if $R$ is true beforehand, it must still be true, so the rule is sound
    
\end{enumerate}

\end{document}
