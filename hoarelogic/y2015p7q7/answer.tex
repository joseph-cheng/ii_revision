\documentclass[12pt]{article}
\usepackage[margin=1.5cm]{geometry}
\usepackage{parskip}
\usepackage{amsmath}
\usepackage{amssymb}
\usepackage{amsfonts}
\usepackage{enumitem}
\usepackage{graphicx}
\usepackage{stmaryrd}
\graphicspath{ {./images/} }


\begin{document}
\begin{enumerate}[label=(\alph*)]

  \item
    Not relevant
\item
  For assertions $P,Q$ and command $C$, we have that $\{P\}C\{Q\}$ holds in separation logic if for all heaps $h_1$ and stacks $s$ that satisfy $P$, if we choose arbitrary $h_F$ disjoint from $h_1$, then $\langle C, \langle h_1 \uplus h_F, s \rangle \rangle \not \leadsto^* \lightning$ (it does not fault), and if $\langle C, \langle h_1 \uplus h_f, s \rangle \rangle$ terminates in some program state $h', s'$, then there exists an $h_1'$ such that $h' = h_1' \uplus h_F$, and then $h_1', s'$ satisfy $Q$.

  \item
    \begin{enumerate}[label=(\roman*)]
        \item
          This triple is not true.

          In order for us to be able to prove it true, we would have to be able to reduce it to something of the form $\{X \mapsto t\}[X]:= 1\{X \mapsto 1\}$

          Since the given precondition ($\{X = 0\}$) has an empty heap, there is no way we can get the precondition $\{X \mapsto t\}$ for some $t$, since there are no allocations in the command. Therefore, by the relative completeness of separation logic, this triple is not true.

          Another argument is that $\{X =0\}[X]:= 1\{X \mapsto 1\}$ requires as a precondition that the heap is empty, so the assignment $[X] := 1$ will fault.

          \item
            Not relevant.
          \item

            Not relevant.
    \end{enumerate}

        
    \end{enumerate}
\end{document}
