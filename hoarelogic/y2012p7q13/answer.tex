\documentclass[12pt]{article}
\usepackage[margin=1.5cm]{geometry}
\usepackage{parskip}
\usepackage{amsmath}
\usepackage{amssymb}
\usepackage{amsfonts}
\usepackage{enumitem}
\usepackage{graphicx}
\usepackage{stmaryrd}
\graphicspath{ {./images/} }


\begin{document}
\begin{enumerate}[label=(\alph*)]

  \item
    We can represent this state space by a single natural number, representing the value of \texttt{N}, since the value of \texttt{N} perfectly describes the state of the program (\texttt{N=1} only if we are in the first line, \texttt{N>1} only if we are in the loop, and \texttt{N =0} only if we are at the end of the program).

    Given this, we have that two states $n,n'$ are related if $n' = n+1$ and $n \neq 0$, or if $n' = 0$ (such that 0 relates to itself [and only itself] for left-totality).

    The only start state is $n=1$.
    
    \item

      We introduce two atomic propositions: $term$ and $zero$. $term$ and $z$ are both only true in the state where $n=0$.

      \begin{enumerate}[label=(\roman*)]

        \item
          We provide the following formula, in CTL:

          $A F(term)$

          This formula means that along all paths from the start state, somewhere along that path we reach a state that terminates.

          This is not true in the given model, since there exists a path $1 \rightarrow 2 \rightarrow \cdots$ forever, and this never reaches state 0 and thus never terminates.

          \item
            We provide the following formula, in CTL:

            $E F(term)$

            This formula means that along some path from the start state, somewhere along that path we reach a state that terminates.

            \item
              We provide the following formula, in CTL:

              $AG (term \implies zero)$

              This formula means that in all reachable states, if we terminate, then zero also holds.

              This does hold, since the only state where $term$ holds is the state for 0, where $zero$ also holds.
          
      \end{enumerate}


        
    \end{enumerate}
\end{document}
