\documentclass[12pt]{article}
\usepackage[margin=1.5cm]{geometry}
\usepackage{parskip}
\usepackage{amsmath}
\usepackage{amssymb}
\usepackage{amsfonts}
\usepackage{enumitem}
\usepackage{graphicx}
\usepackage{stmaryrd}
\graphicspath{ {./images/} }

\begin{document}
    

\begin{enumerate}[label=(\alph*)]

  \item
  \begin{enumerate}[label=(\roman*)]
      
  \item
    We can represent our states by a 10-tuple, where the first 9 elements are from the set $\{0,1,2\}$ representing nought, cross, and empty space respectively. These 9 elements read across in rows, so the fourth element of the tuple is the center-left cell. Finally, the 10th element of the tuple is in the set $\{0,1\}$ representing whose turn is next (0 for nought, 1 for cross).

    \item
      

      The set of initial states are the two states $(0,0,0,0,0,0,0,0,0,0)$ and $(0,0,0,0,0,0,0,0,0,1)$, representing the empty grid, with either player to start.

      \item


        We have that $R(s,s')$ if $s$ is of the form $(s_0, s_1, s_2, s_3, s_4, s_5, s_6, s_7, s_8, p)$, and $s'$ is of the form $s[p / s_i, 1-p / p]$ where $i$ is arbitrary as long as $s_i = 2$, 

        Where $s[x / y, u / v]$ corresponds to replacing occurrences of $y$ with $x$ and $v$ with $u$ simultaneously in $s$.

        \item
          For the state $s = (s_0, s_1, s_2, s_3, s_4, s_5, s_6, s_7, s_8, s_9)$, we have that $Has(i, s_{i-1}) \in L(s)$ holds for all $i \in \{1\ldots9\}$

          We also have that $Start(p) \in s$ if $s_9 = p$ and $2 \in \{s_0, \ldots, s_8\}$
    
  \end{enumerate}

  \item
    We give a CTL formula:

    $(Start(0) \implies E X (A X (E X (A X (E X (A X (E X (A X (E X (Win(0) \vee Draw(0))))))))))) \wedge (Start(1) \implies E X (A X (E X (A X (E X (A X (E X (A X (E X (Win(1) \vee Draw(1)))))))))))$

    Where $Win(p)$ is the compound formula:

    $(Has(1,p) \wedge Has(2,p) \wedge Has(3,p)) \vee$

    $(Has(4,p) \wedge Has(5,p) \wedge Has(6,p)) \vee$

    $(Has(7,p) \wedge Has(8,p) \wedge Has(9,p)) \vee$

    $(Has(1,p) \wedge Has(4,p) \wedge Has(7,p)) \vee$

    $(Has(2,p) \wedge Has(5,p) \wedge Has(8,p)) \vee$

    $(Has(3,p) \wedge Has(6,p) \wedge Has(9,p)) \vee$

    $(Has(1,p) \wedge Has(5,p) \wedge Has(9,p)) \vee$

    $(Has(3,p) \wedge Has(5,p) \wedge Has(7,p))$

    And $Draw(p)$ is the compound formula:

    $\neg (Has(1, 2) \vee \ldots \vee Has(9,2)) \wedge \neg (Win(0) \vee Win(1))$

    This formula holds because there are at most 9 moves in a given game, so the computation tree has only a depth of 9. At each step, we use either $E X$ (if the first player is playing) to say that we can choose a `best' move, and $A X$ (if the second player is playing) to ensure that we consider every possible play that the second player can make.

    We duplicate this for both players.





\end{enumerate}
\end{document}
