\documentclass[12pt]{article}
\usepackage[margin=1.5cm]{geometry}
\usepackage{parskip}
\usepackage{amsmath}
\usepackage{amssymb}
\usepackage{amsfonts}
\usepackage{enumitem}
\usepackage{graphicx}
\usepackage{stmaryrd}
\graphicspath{ {./images/} }


\begin{document}
\begin{enumerate}[label=(\alph*)]

  \item
    \begin{enumerate}[label=(\roman*)]

      \item

        Consider the triple $\{\top\}\texttt{X := X + 1}\{X = X+1\}$.

        This cannot  be proved using Hoare Logic because it is not true, since the computation terminates, and $X$ cannot equal $X+1$. Therefore, by soundness of Hoare logic, it is unprovable.

        \item
          The issue arises when $E$ contains $V$ as a free variable. Therefore, we have the condition that $V \not \in fv(E)$ (this is technically too strong a condition, since $V := V$ is fine, but this still is valid for the question).

          To prove a triple of this form, by the rule of consequence we are simply required to prove $\{E = E\}V := E\{V = E\}$, which we can prove with the assignment rule.

          \item
            $wlp(V:= E, Q) = [E / V]Q$. This holds because this is the weakest condition that guarantees that $Q$ will hold after executing the assignment. This relates to the given Hoare triple in that it holds iff $P \implies [E / V]Q$

            $sp(V:= E, P) = \exists v. [v / V]P \wedge V = [v / V]E$. This holds because if $P$ held before the assignment, it must still hold (with the old value of $V$), and this new value of $V$ must be equal to $E$ (with the corresponding substitution applied). This relates to the given Hoare triple in that it holds iff $sp(V:= E, P) \implies Q$.
            \item
              Weakest preconditions and strongest postconditions can be applied in mechanised program verification by using weakest preconditions to show that the given precondition for a program implies the weakest precondition calculated from the given postcondition, as well as proving that loop invariants really are invariant.

              Strongest postconditions can be used to provide some kind of semantics for a language, letting us derive what the behaviour of a given program might be.

              \item
                The two axioms are as follows:

                \[
                  \frac{}{\{[E / X]P\}X := E\{P\}}
                .\] 

                \[
                  \frac{}{\{P\}X:= E\{\exists x. [x / X]P \wedge X = [x / X]E\}}
                .\] 

                The Hoare axiom is true because if $P$ holds after some assignment, it must have also held before the assignment, but instead of using $X$, it must hold for $E$ instead.

                The Floyd axiom holds because if $P$ holds before some assignment, then after the assignment $P$ must also hold, but using the old value of $X$, and $X$ must now equal its newly assigned value.

                \item
                  Suppose Hoare's axiom holds.

                  We are then required to prove that $\{P\}X:=E\{\exists x[x / X]P \wedge X = [x / X]E\}$ holds.

                  We know that $\{\exists x.[x / X]P \wedge X = x\}X :=E\{\exists x.[x / X]P \wedge X = [x / X]E\}$ holds by Hoare's axiom.


                We have that $P \implies (X = X) \wedge P \implies (X = X) \wedge [X / X]P \implies \exists x. [x / X]P \wedge X = x$

                And therefore, we have that $\{P\}X:=E \{\exists x.[x / X]P \wedge X = [x / X]E\}$ holds.

        
    \end{enumerate}
        
    \end{enumerate}
\end{document}
