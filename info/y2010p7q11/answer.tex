\documentclass[12pt]{article}
\usepackage[margin=1.5cm]{geometry}
\usepackage{parskip}
\usepackage{amsmath}
\usepackage{amssymb}
\usepackage{amsfonts}
\usepackage{enumitem}
\usepackage{graphicx}
\usepackage{stmaryrd}
\graphicspath{ {./images/} }


\begin{document}
\begin{enumerate}[label=(\alph*)]
  \item

    \begin{enumerate}[label=(\roman*)]
      \item
        The mutual information $I(X;Y)$ is defined as follows:

        $\sum_{x \in X} \sum_{y \in Y} p(x,y) \log_2(\frac{p(x,y)}{p(x)p(y)})$
      \item

        In the case that $X$ and $Y$ are independent, mutual information goes to 0 because no information is learned about the other upon observing a value. In the sum, this is because $p(x,y) = p(x)p(y)$ so the sum goes to 0.

      \item
        If $X$ and $Y$ are perfectly correlated with each other, this means that $p(x) = p(y)$, and thus that $p(x,y) = p(x)$

        Therefore, the mutual information is:

        $\sum_{x \in X}p(x) \log_2(\frac{p(x)}{p(x)^2})$

        Which is the same as $-\sum_{x \in X}p(x)\log_2 p(x) = H(X)$.

      \item
        This is equivalent to (iii), so $H(X)$.






        
    \end{enumerate}

  \item
    The information gained from observing event $i$ is $\log_2 p_i$, so the sum of information gained from observing each of the events separately and in any order is $\sum_{i=1}^N \log_2 p_i$

    The information gained from observing all independent events is given by $\log_2 (\sum_{i=1}^N p_i) = \sum_{i=1}^N \log_2 p_i$, so we are done.

  \item
    Not relevant.


  \item
    Not relevant.

  \item
    Not relevant.
\end{enumerate}
    
\end{document}

