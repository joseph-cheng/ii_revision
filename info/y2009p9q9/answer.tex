\documentclass[12pt]{article}
\usepackage[margin=1.5cm]{geometry}
\usepackage{parskip}
\usepackage{amsmath}
\usepackage{amssymb}
\usepackage{amsfonts}
\usepackage{enumitem}
\usepackage{graphicx}
\usepackage{stmaryrd}
\graphicspath{ {./images/} }


\begin{document}

\begin{enumerate}[label=(\alph*)]
  \item
    \begin{enumerate}[label=(\roman*)]

      \item
        This is just $256 \cdot \frac{1}{256} \cdot \log_2 {\frac{1}{256}} = 8$ bits.

      \item
        Since all humans are mammals, this has 0 entropy (all humans are in the same group).

      \item
        We just apply the entropy formula:

        $\frac{1}{4}\log_2 \frac{1}{4} + \frac{1}{4}\log_2 \frac{1}{4} + \frac{1}{2}\log_2 \frac{1}{2} = \frac{3}{2}$ bits.

      \item
        By definition, half the population is below the median, and half are above, so this has entropy $\frac{1}{2}\log_2 \frac{1}{2} + \frac{1}{2}\log_2 \frac{1}{2} = 1$ bit.

        
    \end{enumerate}

  \item
    \begin{enumerate}[label=(\alph*)]
      \item
        the Kullback-Leibler distance of two discrete probability distributions $p(x)$ and $q(x)$ is defined as follows:

        $D_{KL}(p(x), q(x)) = \sum_{x \in \text{dom}(p)} p(x) \log_2 \frac{p(x)}{q(x)}$

      \item
        The KL will be 0 when the code is most effective (i.e. when $p(x) = q(x)$). As the two probability distributions get further apart, the KL increases, implying a worse effectiveness. Precisely, the KL tells us the number of additional bits we need on average if we are actually using $p(x)$.

      \item
        The KL violates the symmetry axiom of distance metrics.

      \item
        If there are values where $q(x)$ is 0, and $p(x)$ is \textit{not} zero, then the KL becomes undefined, or goes to infinity. If $p(x)$ and $q(x)$ are both zero, then this does not imply it goes to infinity.
        
    \end{enumerate}

  \item
    \begin{enumerate}[label=(\roman*)]
      \item
        The Nyquist theorem tells us that we need to sample at twice the highest frequency, or 2 KHz. For one second of recording, this results in 2000 elements.

      \item
        Not relevant.

    \end{enumerate}
  \item
    Not relevant
    
\end{enumerate}
    
\end{document}
