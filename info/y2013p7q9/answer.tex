\documentclass[12pt]{article}
\usepackage[margin=1.5cm]{geometry}
\usepackage{parskip}
\usepackage{amsmath}
\usepackage{amssymb}
\usepackage{amsfonts}
\usepackage{enumitem}
\usepackage{graphicx}
\usepackage{stmaryrd}
\graphicspath{ {./images/} }


\begin{document}
\begin{enumerate}[label=(\alph*)]
    \item
        \begin{enumerate}[label=(\roman*)]
            \item
                We can first ask `Is the symbol E?'.

                Then, `Is the symbol B?'

                Then, `Is the symbol C?'

                Then, arbitrarily, `Is the symbol A?'

            \item
                At each question, we halve our entropy. This means that each question removes on average 1 bit of entropy, which is maximal (any more entropy and we violate the Source Coding Theorem), and thus this is optimal.

            \item
                The probability we ask 1 question is $\frac{1}{2}$

                The probability we ask 2 questions is $\frac{1}{4}$

                The probability we ask 3 questions is $\frac{1}{8}$

                The probability we ask 4 questions is $\frac{1}{8}$

                Overall, the expected number of questions we ask is $\frac{1}{2} + \frac{1}{2} + \frac{3}{8} + \frac{1}{2} = \frac{15}{8}$.

            \item
                \begin{tabular}{cc}
                    Symbol&Code\\
                    \hline
                    A&1110\\
                    B&10\\
                    C&110\\
                    D&1111\\
                    E&0
                    
                \end{tabular}

                This is uniquely decodable because it is never ambiguous as to which codeword we mean in a sequence of bits. This comes from the prefix property, where no codeword is a prefix of another, and so we never read the prefix of some code and misunderstand it as the whole code of another symbol.

            \item
                These bits correspond to whether or not we say yes/no for each question in (i). For example, the symbol C corresponds to saying `no' to the first two questions and `yes' to the third. This applies to all of the codewords.


            
        \end{enumerate}

    \item
        Not relevant

    \item

        The Shannon-Hartley theorem tells us that capacity is given by the following:

        \[
            C = \omega \log_2 (1 + \frac{P}{N_0 \omega})
        \] 

        In our case, we get about $10^7$ bits.

        When the signal to noise ratio is about four times better, the capacity is then:

        \[
            C = \omega \log_2 (1 + 4\frac{P}{N_0 \omega})
        \] 

        Since 1 is small in comparison, this is approximately $\omega \log_2(4 \frac{P}{N_0 \omega}) = \omega(2 + \log_2 \frac{P}{N_0 \omega})$, so the capacity is  increased by a factor of $2 \cdot 10^6$, to around $1.2 \cdot 10^7$ bits per second.
    



    
    \end{enumerate}
\end{document}
