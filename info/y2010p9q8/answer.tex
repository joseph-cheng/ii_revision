\documentclass[12pt]{article}
\usepackage[margin=1.5cm]{geometry}
\usepackage{parskip}
\usepackage{amsmath}
\usepackage{amssymb}
\usepackage{amsfonts}
\usepackage{enumitem}
\usepackage{graphicx}
\usepackage{stmaryrd}
\graphicspath{ {./images/} }


\begin{document}
\begin{enumerate}[label=(\alph*)]
  \item
    Suppose we have an unlimited size alphabet $A$. Then, we number the alphabet with a bijection with the natural numbers $\{1,2,3,...\}$ called $B$.

    Then, we assign probabilities to $A$ as follows: $P(a) = 2^{-B(a)}$

    Note that this is always non-zero, as required.

    Furthermore, we have that $\sum_{a \in A} P(a) = \sum_{i=1}^\infty 2^{-i} = 1$, as a known limit, so the probabilities sum to 1 as required.

    The entropy of this source is $-\sum_{a \in A}P(a) \log_2 P(a) = \sum_{i=1}^\infty 2^{-i}\log_2(2^i) = 2$ bits, as a known limit given in the question.

    Therefore, we know by the Source Coding Theorem that a code exists for this source with average codeword length no more than 2 bits.

    For character $a$, with index $n = B(a)$, we provide the codeword of $n-1$ leading 0s, followed by a 1, which generates a uniquely decodable prefix code.

    The average codeword length even at unlimited size is given by $\sum_{i=1}^\infty i2^{-i} = 2$.

  \item
    Not relevant

  \item

    To define the bit rate for this channel, we must consider the information rate given by each frequency in the frequency range. This is given by $\delta \omega \log_2(1 + SNR(\omega))$ for a given $\omega$ and small change $\delta \omega$.

    Therefore, the bit rate for this channel is given by the following sum/integral:

    \[
      \int_{\omega_1}^{\omega_2} \log_2(1 + SNR(\omega)) \mathrm{d}\omega
    \] 

  \item
    Not relevant.









        
\end{enumerate}
\end{document}
