\documentclass[12pt]{article}
\usepackage[margin=1.5cm]{geometry}
\usepackage{parskip}
\usepackage{amsmath}
\usepackage{amssymb}
\usepackage{amsfonts}
\usepackage{enumitem}
\usepackage{graphicx}
\usepackage{stmaryrd}
\graphicspath{ {./images/} }


\begin{document}
\begin{enumerate}[label=1.\arabic*]
  \setcounter{enumi}{1}
  \item

    For a binary symmetric channel with noise level $f$, we get an error if 2 or more bits are corrupted in the repetition. This means that as long as we get 0 or 1 bit errors, we are fine.

    The probability we get 2 bit errors is $3f^2(1-f)$. The probability we get 3 bit errors is $f^3$, so the probability we get an error in decoding is $f^3 + 3f^2(1-f) = 3f^2 - 2f^3$

    When $0 \leq f \leq 1$, we see that $3f^2 - 2f^3 \leq f$, and so the error probability is reduced by using $R_3$.

  \item
    \begin{enumerate}
        \item

          For a binary symmetric channel with noise level $f$, we get an error when the majority of bits have bit errors, which is when we have $\frac{N+1}{2}$ bit errors, for code $R_N$.

          The probability that this happens can be modelled binomially. The probability we get exactly $i$ bit errors is given by $\binom{n}{i} f^i (1-f)^{N-i}$, so the probability we get an error in decoding is:

          \[
            p_b = \sum_{i= \frac{N+1}{2}}^N \binom{N}{i} f^i (1-f)^{N-i}
          .\] 

          As required.

        \item
          If $f=0.1$, the biggest term will be the one with the least multiple of $f$, i.e. the term where $i = \frac{N+1}{2}$, the minimum value of $i$

          The second biggest term will have its binomial coefficient decrease by a factor of $\frac{N-1}{N+3}$, and the rest will be equal to $f^{\frac{N+3}{2}}(1-f)^{\frac{N-3}{2}}$, so a decrease of $\frac{f}{1-f} = \frac{1}{9}$. Therefore, the next biggest term will be $\frac{N-1}{9(N+3)}$ smaller.

        \item
          Stirling's approximation tells us $x! \approx x^xe^{-x}$

          So for $\binom{N}{\frac{N+1}{2}}$, we approximate to 

          \[
          \frac{N^Ne^{-N}}{\frac{N-1}{2}^{\frac{N-1}{2}} e^{-\frac{N-1}{2}} \cdot \frac{N+1}{2}^{\frac{N+1}{2}} e^{-\frac{N+1}{2}}}
          .\] 

          We assume this dominates, so multiplying by $f^{\frac{N+1}{2}}(1-f)^{\frac{N-1}{2}}$, we get our approximate error.

        \item
          Assuming $f=0.1$, we iteratively solve:

\begin{verbatim}
import math
def f(N):
    numerator = N**N * math.exp(-N)
    denominator = ((N-1)/2)**((N-1)/2) * math.exp(-(N-1)/2) * ((N+1)/2)**((N+1)/2) * math.exp(-(N+1)/2)
    frac = numerator/denominator
    f = 0.1
    return frac * (f**((N+1)/2)) * (1-f)**((N-1)/2)

for N in range(1, 100):
    error = f(N)
    if error < 10e-15:
        print(N)
        break
\end{verbatim}

This gives us an answer of 61.

    \end{enumerate}

  \item
  \item
  \item
    \begin{enumerate}
      \item
        A (7,4) Hamming code can correct any 1-bit errors (and of course 0-bit errors).

        Therefore, we get a block error when we get 2 or more bit errors.

        Just like in exercise 1.3, we get:

        \[
          p_B = \sum_{i=2}^7 \binom{7}{i} f^i(1-f)^{7-i}
        .\] 

        This polynomial will have leading order $2$, and this will only appear when $i=2$, so the leading order is $\binom{7}{2}f^2 = 21f^2$.

      \item
        In Hamming codes, a two-bit error (which is what we consider, for leading order) will result in a resulting decoded codeword with 3 errors, of length 7 (since we incorrectly identify the troublesome bits, and flip an innocent bit).

        By symmetry, the probability of any of these 3 bit flips is equally likely, so the probability that a given bit is flipped will be $\frac{3}{7}$. Therefore, our probability of error is $\frac{3}{7} \cdot 21 f^2 = 9f^2$.
      \item


    \end{enumerate}


      \item

      \item
      \item
      \item

        In 14 bits, the number of two errors is $\binom{14}{2} = 91$

        Since we only have 6 parity bits, this gives us $2^6 = 64$ syndromes.

        Therefore, no $(14,8)$ code can exist to correct all two errors, since we do not have enough syndromes to capture every two error.

      \item
      \item
        We only consider leading order.

        This requires us to get two bit errors in our $R_3$ codes, so we get $p_b(R_3) \cdot p_b(R_3)$ which has leading order $9f^4$

        On the other hand, $R_9$ is dominated by leading order $126f^5$, so is asymptotically better.




        
\end{enumerate}
\end{document}
