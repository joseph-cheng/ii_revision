\documentclass[12pt]{article}
\usepackage[margin=1.5cm]{geometry}
\usepackage{parskip}
\usepackage{amsmath}
\usepackage{amssymb}
\usepackage{amsfonts}
\usepackage{enumitem}
\usepackage{graphicx}
\usepackage{stmaryrd}
\graphicspath{ {./images/} }


\begin{document}
\begin{enumerate}[label=(\alph*)]
  \item
    \begin{enumerate}[label=(\roman*)]

      \item

        A Hamming Code has 3 parity bits and 4 data bits. The maximum possible rate of information transmission is given when we have no errors, in which case we receive 4 bits of data and have to transmit 7, so the maximum possible rate of information transmission is $\frac{4}{7}$.

      \item
        The data bits are combined with an XOR operation to build the syndromes. This operator is applied before transmission (to compute the parity bits), but upon reception to actually compute the syndrome of a particular block.
        
    \end{enumerate}

  \item
    Not relevant.

  \item
    Not relevant.
        
\end{enumerate}
\end{document}
