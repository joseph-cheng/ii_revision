\documentclass[12pt]{article}
\usepackage[margin=1.5cm]{geometry}
\usepackage{parskip}
\usepackage{amsmath}
\usepackage{amssymb}
\usepackage{amsfonts}
\usepackage{enumitem}
\usepackage{graphicx}
\usepackage{stmaryrd}
\graphicspath{ {./images/} }


\begin{document}
\begin{enumerate}[label=(\alph*)]

  \item
    \begin{enumerate}[label=(\roman*)]

      \item
        Since $Y$ takes a different value for each value of $X$, $H(Y) = H(X)$, so 8 bits

      \item
        Since $Y$ is completely determined by the value of $X$, $H(Y|X) = 0$ bits.

      \item
        0 bits, since $H(X|Y) = H(Y|X) + H(Y) - H(X) = 0 + 8 - 8 = 0$.

      \item
        $H(X,Y) = 8$ bits, since $X$ and $Y$ are perfectly correlated.

      \item
        In this case, we know that $H(Y) < H(X)$, since we have lost entropy by encoding some values of $X$ as the same.

      \item
        We know that $H(X|Y) > 0$, since $Y$ no  longer tells us all the information we need to know $X$.

        
    \end{enumerate}

  \item
    \begin{enumerate}[label=(\roman*)]

      \item
        Nyquist's theorem tells us that we must sample at most double the highest frequency observed in a signal, so we need to sample at $2W$ Hz. This corresponds to $2TW$ samples in a duration of time $T$

      \item
        Since we know the bandwidth is limited at $W$ Hz, the signal cannot do anything unexpected between samples, or else it would violate the bandwidth restriction. Therefore, the signal is completely determined by the samples.

    \end{enumerate}

  \item
    Not relevant.

  \item
    Not relevant.

  \item
    Not relevant.
        
    \end{enumerate}
\end{document}
