\documentclass[12pt]{article}
\usepackage[margin=1.5cm]{geometry}
\usepackage{parskip}
\usepackage{amsmath}
\usepackage{amssymb}
\usepackage{amsfonts}
\usepackage{enumitem}
\usepackage{graphicx}
\usepackage{stmaryrd}
\graphicspath{ {./images/} }


\begin{document}
\begin{enumerate}[label=(\alph*)]
  \item

  \begin{enumerate}[label=(\roman*)]
    \item
      The most efficient sequence of questions would be as follows:

      Is it A?

      Is it B?

      Is it C?

      Is it D?

      Is it E?

      Is it F?

      Is it G?

    \item
      We can claim that each of the proposed questions is maximally informative because each of the questions reduces the remaining uncertainty by 1 bit, and therefore we maximally reduce the remaining entropy with each question.

    \item
      We calculate:

      $\frac{1}{2}  + \frac{2}{4} + \frac{3}{8} + \frac{4}{16} + \frac{5}{32} + \frac{6}{64} + \frac{7}{128} + \frac{7}{128} = \frac{127}{64}$.

    \item
      The entropy of the above symbol set is $\sum_{x \in S} p(x)\log_2 p(x)$, which results in the same sum as (iii), giving us an entropy of $\frac{127}{64}$ bits.

    \item
      We give the following code:

\begin{verbatim}
A: 1
B: 01
C: 001
D: 0001
E: 00001
F: 000001
G: 0000001
H: 0000000
\end{verbatim}

It is uniquely decodable because no code can be interpreted as the concatenation of codes for other symbols. It has the prefix property because no code is the prefix of another.

\item
  The 0 bit represents a `no' answer, and the 1 bit represents a `yes' answer, if we were to ask the questions in (i).
      
  \end{enumerate}

\item
  Not relevant.

\item
  Not relevant.

\item
  Not relevant.
        
    \end{enumerate}
\end{document}
