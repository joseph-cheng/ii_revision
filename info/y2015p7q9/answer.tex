\documentclass[12pt]{article}
\usepackage[margin=1.5cm]{geometry}
\usepackage{parskip}
\usepackage{amsmath}
\usepackage{amssymb}
\usepackage{amsfonts}
\usepackage{enumitem}
\usepackage{graphicx}
\usepackage{stmaryrd}
\graphicspath{ {./images/} }


\begin{document}
\begin{enumerate}[label=(\alph*)]
  \item
    Not relevant
  \item
    \begin{enumerate}[label=(\roman*)]
      \item

        The conditions that would maximise the channel capacity are when the input source is uniformly distributed, and when either $p=0$ or $p=1$ (in both cases, we are able to perfectly reconstruct the original message).

      \item
        Such a scheme fails when we have any more than 1 bit flip in 7 bits.

        The probability we get 0 bit flips is $(1-p)^7$, and the probability we get a single bit flip is $7p(1-p)^6$, so the probability that we get more than 1 bit flip is $P_e = 1 - (1-p)^7 - 7p(1-p)^6$.

      \item
        The repetition scheme fails when we get $m+1$ bit flips in $2m+1$ bits.

        The probability this happens can be derived if we model this as a binomial distribution $X \sim B(2m+1, p)$, where we need $P(X \geq m+1)$.

        $P_e = P(X \geq m) = \sum_{i=2m+1}^{m+1} \binom{2m+1}{i} p^i(1-p)^{2m+1-i}$
        
    \end{enumerate}

  \item
    Not relevant
        
    \end{enumerate}
\end{document}
