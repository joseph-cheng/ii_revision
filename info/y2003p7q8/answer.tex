\documentclass[12pt]{article}
\usepackage[margin=1.5cm]{geometry}
\usepackage{parskip}
\usepackage{amsmath}
\usepackage{amssymb}
\usepackage{amsfonts}
\usepackage{enumitem}
\usepackage{graphicx}
\usepackage{stmaryrd}
\graphicspath{ {./images/} }


\begin{document}
\begin{enumerate}[label=(\alph*)]

  \item
     Not relevant?

   \item
     The average codeword lengths for the two codes are:

     Code 1: $2p(a) + 2p(b) + 2p(c) + 2p(d)$

     Code 2: $p(a) + 2p(b) + 3p(c) + 3p(d)$

     We require that code 2 is better than code 1, so $p(a) + 2p(b) + 3p(c) + 3p(d) < 2p(a) + 2p(b) + 2p(c) + 2p(d)$

     This simplifies to $p(a) > p(c) + p(d)$

     Since $p(a) \geq p(b) \geq p(c) \geq p(d)$, we require that $p(a) > \frac{1}{3}$, or else $p(c) + p(d) \geq p(a)$.

     After $p(a) > 0.5$, any choice of $p(b), p(c), p(d)$ will satisfy the requirements.

   \item
     By the Source Coding Theorem, we know that the average length per symbol of the binary code will always be greater than or equal to $H(X)$, if $X$ is the source variable.

     If $X$ is our alphabet consisting of $m$ equiprobable symbols, then its entropy is:

     $H(X) = -\sum_{i=1}^m \frac{1}{m} \log_2 \frac{1}{m} = \log_2 m$

     Therefore, $H(X) = \log_2 m$, and since the average length of any code is $\geq H(x)$, it is $\geq \log_2 m$.
    \end{enumerate}
\end{document}
