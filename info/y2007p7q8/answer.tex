\documentclass[12pt]{article}
\usepackage[margin=1.5cm]{geometry}
\usepackage{parskip}
\usepackage{amsmath}
\usepackage{amssymb}
\usepackage{amsfonts}
\usepackage{enumitem}
\usepackage{graphicx}
\usepackage{stmaryrd}
\graphicspath{ {./images/} }


\begin{document}
\begin{enumerate}[label=(\alph*)]
  \item
    \begin{enumerate}[label=(\roman*)]

      \item
        A distribution is given below:

\begin{verbatim}
A: 1/2
B: 1/8
C: 1/8
D: 1/16
E: 1/16
F: 1/16
G: 1/16
\end{verbatim}

\item
  Each question had an uncertainty of 1 bit, since this sequence is optimal.

\item
  The entropy of this alphabet is 2.25 bits, through the definition of entropy.

\item
\begin{verbatim}
A: 0
B: 111
C: 110
D: 1011
E: 1010
F: 1001
G: 1000
\end{verbatim}


\item
  The average coding rate $R$ is given by $\frac{1}{2} + 2 \cdot \frac{3}{8} + 4 \cdot \frac{4}{16} = 2.25$ bits.

\item
  A more efficient code could not be developed because of the Source Coding Theorem, which tells us that we cannot losslessly compress a signal below its entropy. Since the average coding rate is equal to the entropy of the source, no more efficient code can exist.
        
    \end{enumerate}

  \item

    The information capacity over any tiny bandwidth is given by $\delta \omega \log_2(1 + SNR(\omega))$, where the frequency is $\omega$.

    To find the capacity, we need to sum up the information capacity given by each tiny bandwidth, which gives us the following integral:

    \[
      \int_{\omega_1}^{\omega_2} \log_2(1 + SNR(\omega)) \mathrm{d}\omega
    .\] 

  \item
    Not relevant.

  \item
    Not relevant.

        
\end{enumerate}
\end{document}
