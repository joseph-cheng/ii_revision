\documentclass[12pt]{article}
\usepackage[margin=1.5cm]{geometry}
\usepackage{parskip}
\usepackage{amsmath}
\usepackage{amssymb}
\usepackage{amsfonts}
\usepackage{enumitem}
\usepackage{graphicx}
\usepackage{stmaryrd}
\graphicspath{ {./images/} }


\begin{document}
\begin{enumerate}[label=(\alph*)]

  \item
    (Second half of this question is not relevant).

    We know that any real continuous channel will have Gaussian noise (or else we can transmit with infinite precision and transmit infinite data per symbol).


    Since the noise is Gaussian, and the output is Gaussian, we require that the input is Gaussian for maximum mutual information.

    Therefore, the capacity of the channel is given by $H(Y) - H(Y|X)$ with $X,Y$ Gaussian.

    Furthermore, we know that $H(Y|X)$ must be $H(N)$ where $N$ is the noise, because that is all the uncertainty that is added.

    If $Y = X + N$ and $X \sim N(\mu_X, \sigma_X^2)$ and $N \sim N(\mu_N, \sigma_N^2)$, then $Y \sim N(\mu_X + \mu_N, \sigma_X^2 + \sigma_N^2)$

    So, $H(Y) =\frac{1}{2}\log_2 (2\pi e (\sigma_X^2 + \sigma_N^2))$

    And $H(Y|X) = \frac{1}{2}\log_2(2\pi e (\sigma_N^2))$

    So $C = \frac{1}{2}\log_2 (2\pi e (1 + \frac{\sigma_X^2}{\sigma_N^2})$

   $C = \frac{1}{2}\log_2 2\pi e (1 + \frac{S}{N})$

   If we are transmitting a bandwidth limited signal, let the signal transmit between frequencies $\omega_1$ and $\omega_2$ ($\omega_1 = 0$ for lowpass, e.g.), with bandwidth $\omega = \omega_2 - \omega_1$

   Assuming that the signal's signal to noise ratio does not change with frequency, the capacity of the signal in  information per second (as opposed to information per symbol) becomes $\omega \log_2 2\pi e (1 + + \frac{S}{N})$ because we must sample at $2\omega$.

   \item
     This code must satisfy the Kraft-MacMillan inequality, which says that:

     $\sum_{i=1}^N n_i^{-1} \leq 1$

     \item
       Not relevant.


        
    \end{enumerate}
\end{document}
