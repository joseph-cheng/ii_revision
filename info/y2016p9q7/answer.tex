\documentclass[12pt]{article}
\usepackage[margin=1.5cm]{geometry}
\usepackage{parskip}
\usepackage{amsmath}
\usepackage{amssymb}
\usepackage{amsfonts}
\usepackage{enumitem}
\usepackage{graphicx}
\usepackage{stmaryrd}
\graphicspath{ {./images/} }


\begin{document}
\begin{enumerate}[label=(\alph*)]
        
  The Shannon-Hartley Theorem tells us that the capacity of such a channel, in bits per second, is given by:

  $C = B\log_2(1 + \frac{S}{N})$, where $\frac{S}{N}$ is the signal-to-noise ratio, and $B$ is our bandwidth in Hertz.

  Since the signal to noise power ratio is 30 decibels (corresponding to 3 bels), the signal to noise ratio is $10^3 = 1000$

  Furthermore, we are given that the bandwidth is $10^7$

  So overall, our capacity is $C = 10^7 \log_2(1001) = 9.97^7$ bits per second (3 s.f.).

\item
  Not relevant

\item
  Not relevant
\end{enumerate}
\end{document}
