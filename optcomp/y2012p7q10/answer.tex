\documentclass[12pt]{article}
\usepackage[margin=1.5cm]{geometry}
\usepackage{parskip}
\usepackage{amsmath}
\usepackage{amssymb}
\usepackage{amsfonts}
\usepackage{enumitem}
\usepackage{graphicx}
\usepackage{stmaryrd}
\graphicspath{ {./images/} }


\begin{document}
\begin{enumerate}[label=(\alph*)]
        
  \item

    The three dataflow anomalies are $UR$, $DD$, and $DU$, corresponding to use-before-defined, write-write, and dead code.

    Use-before-defined is an anomaly because a variable is used before it is defined, so its value will not be defined and behaviour will be unexpected, e.g.:

\begin{verbatim}
void foo() {
  int x;
  print x;
}
\end{verbatim}

Write-write is an anomaly because the first write is useless, we should only need to write a variable once before it is referenced, e.g.:

\begin{verbatim}
void foo() {
  int x = 1;
  int x = 2;
  print x;
}
\end{verbatim}

Dead code is an anomaly because the effects of the write are not observable:

\begin{verbatim}
void foo() {
  int x = 1;
  return;
}
\end{verbatim}

After annotation, we consider pairs by considering two adjacent annotations of a particular variable, where `adjacent' means along any possible path in a dataflow graph, some of which may not be possible in an actual execution of the program.

\item
\begin{verbatim}
void foo() {
  int x = 0, y, z = 1;

  if (x == 0) {
    y = 5;
    z = 2; // DD
  }
  print y; // UR

  x = bar(); // DU
  if (baz()) {
    print z;
    print x;
  }
}
\end{verbatim}

The UR only exists because of the aforementioned approximation, $x$ is always 0 on entry, so the if is always entered, and therefore we always define $y$ before the print, so it is not programmer error.

The DD is not programmer error because the \texttt{z} might be defined at the start of the program for stylistic reasons in the code.

The DU is not programmer error because we might only want to print the variables under some condition, and not otherwise.

\item
  A must-anomaly is a data-flow anomaly that occurs on all paths to a particular program point, whereas a may-anomaly is a data-flow anomaly that occurs on at least one, but not all, paths to a particular program point.

  The previous definition corresponds to may-anomalies.

  The program in part (b) contains no must-anomalies.

\item
  We use the following dataflow equations:

  \[
    may-ur(n) = \left(\bigcup_{s \in succ(n)} may-ur(s)\right) \setminus kill(n) \cup gen(n)
  \]

  \[
    must-ur(n) = \left(\bigcap_{s \in succ(n)} must-ur(s)\right) \setminus kill(n) \cup gen(n)
  \]

  Where the $kill$ and $gen$ sets for a node $n$ are the sets of variables the node kills (defines) and generates (references) respectively.

  These can be used to generate the given compiler messages as follows:

  If, for some scope with entrance and exit at $m$ and $n$ respectively, we have that $x \in may-ur(m)$ and $x \not \in may-ur(n)$, then we can issue ``variable $x$ may be read before being set''.

  Likewise, if $x \in must-ur(m)$ and $x \not \in must-ur(n)$, then we can issue `variable $x$ is definitely read before being set''

  To solve these data-flow equations, sets should be initialised to $\{\}$ for $may-ur$ and $U$ for $must-ur$ (except for the exit points, which should be set to $\{\}$.

\item
  If we have indirect assignments to address-taken variables, then whenever we write to a dereferenced pointer, we could be writing to any address-taken variables, so the above analyses must be modified  to kill all address-taken variables for $may-ur$, and no address-taken variables in $must-ur$. Similarly, we must generate all address-taken variables when we assign from a dereferenced pointer in $may-ur$, and generate no address-taken variables when we assign from a dereferenced pointer in $must-ur$.

\item
  The claim here is that if the loop iterates forever, then we never read $x$, so the wording does not correlate with the execution of the program. We could change the reported error message, and wording, to instead say that the usage of variable $x$ on line ... is done without any definition of $x$. This allows us to avoid making claims on the execution behaviour of the program, which we cannot necessarily know.


\end{enumerate}
\end{document}
