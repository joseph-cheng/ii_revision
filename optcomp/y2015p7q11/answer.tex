\documentclass[12pt]{article}
\usepackage[margin=1.5cm]{geometry}
\usepackage{parskip}
\usepackage{amsmath}
\usepackage{amssymb}
\usepackage{amsfonts}
\usepackage{enumitem}
\usepackage{graphicx}
\usepackage{stmaryrd}
\graphicspath{ {./images/} }


\begin{document}
\begin{enumerate}[label=(\alph*)]
    \item

        This is a forwards-propagating analysis

        We construct the two mutually recursively defined sets:

        $in-reaching(n) = \bigcup_{p \in pred(n)} out-reaching(p)$

        $out-reaching(n) = in-reaching(n) \setminus kill(n) \cup gen(n)$

        Where $kill(n)$ is the set of instructions that $n$ kills, i.e. all of the instructions that assign to the same variable that $n$ assigns to, and $gen(n)$ is the set of instructions that $n$ generates, i.e. itself if it assigns to a variable.

        We then take $reaching = in-reaching$:

        $reaching(n) = \bigcup_{p \in pred(n)} (reaching(p) \setminus kill(p) \cup gen(p))$

        An algorithm for computing this function might look as follows:

\begin{verbatim}
for i=1 to n do reaching[i] = {}
while (reaching[] changes) {
  for i=2 to n do {
    reaching[i] = big_union(p in pred(i), reaching(p) \ kill(p) U gen(p))
  }
}
\end{verbatim}

This algorithm/formulation provides an over-approximation of semantic reaching definitions, since we consider syntactic control flow as opposed to semantic control flow.

For example:

\begin{verbatim}
if ((x +1) * (x+1) == x*x + 2*x + 1) {
  y = 1
} else {
  y = 2
}
print y
\end{verbatim}

In this program, the definition $y=2$ should not reach $print(y)$, since it never gets executed, but in our algorithm we would mark it as reaching $print(y)$.

As long as we keep this in mind, we can still use this safely, by ensuring that our future transformation relies on the fact that instructions \textit{not} in $RD(i)$ definitely do not reach instruction $i$, and not the inverse.

\item

    This analysis is flow-sensitive. The analysis relies on control flow, and the order in which assignments execute, so a flow-insensitive analysis of this kind would be nonsensical.

\item
    Suppose an instruction $I$ reads from operand $y$, and our reaching-definitions analysis shows us that every definition that reaches $I$ and writes to $y$ is of the form $y=k$, for some constant $k$.

    This means, we can replace the use of operand $y$ with constant $k$, which gives us constant propagation.

\item

    The following code would be generated:

\begin{verbatim}
int t,r,x;
x = read();
if (x > 91) t=7; else t=6;
r = t/2;
return r+39
\end{verbatim}

We are unable to do any constant propagation!

This is because both definitions $t=7$ and $t=6$ reach $r=t / 2$.

This loses optimality, however, since assuming that division between two integers is floor division, then $t / 2$ is always 3, regardless of whether $t$ is 6 or 7.

So, the source of the information loss is not considering the results of each branching path: doing so would make us realise that each path results in $r=3$ (and presumably we then also optimise to return $3 + 39$

\item
    In SSA form, we only ever assign to a variable once. This means that every assignment that occurs before an instruction will reach that instruction.

    Furthermore, each instruction in $RD(i)$ will be the only instruction that assigns to its variable.





    
\end{enumerate}
\end{document}
