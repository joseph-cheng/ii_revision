\documentclass[12pt]{article}
\usepackage[margin=1.5cm]{geometry}
\usepackage{parskip}
\usepackage{amsmath}
\usepackage{amssymb}
\usepackage{amsfonts}
\usepackage{enumitem}
\usepackage{graphicx}
\usepackage{stmaryrd}
\graphicspath{ {./images/} }


\begin{document}
\begin{enumerate}[label=(\alph*)]

  \item

    \begin{enumerate}[label=(\roman*)]

      \item

        Strictness analysis is a form of abstract interpretation used for pure, functional languages, where we abstractly interpret expressions as values from the set $2 = \{0,1\}$, where 0 means guaranteed non-termination, and 1 means possible termination.

        Then, a $k$-argument, 1-result function $\mathbb{Z}^k \rightarrow \mathbb{Z}$ (assuming integers here, but could be any domain/range) as $2^k \rightarrow 2$, where given the termination behaviour of its arguments, we get the termination behaviour of the function call.

        Built-in functions are given an abstract interpretation through manual labelling. User-defined functions are given abstract meanings through composition of abstract interpretations of the built-in functions (or perhaps user-defined ones we have already calculated).

        We require some extra work if we allow for recursion in our language, since we cannot simply compute our abstract function interpretations through composition, we have to compute a fixed point instead, by initialising the abstract interpretation(s) to never terminate, and then finding the least fixed point through iteration.

        This analysis allows us to determine which functions are strict in particular arguments. A function is strict in a particular argument if, when that argument does not terminate but all other arguments may terminate, the function does not terminate, i.e. when $f^{\#}(1,\ldots,1,0,1,\ldots,1) = 0$. This allows us to replace the use of this parameter with CBV semantics instead of call-by-name or call-by-need semantics. Using CBV semantics instead of call-by-name semantics means we might avoid having to re-evaluate the argument multiple times, and using CBV semantics instead of call-by-need semantics means we get better asymptotic space complexity because we don't need to carry around a closure, and we also get better cache affinity.

        \item
          \begin{enumerate}[label=(\roman*)]

            \item
              We abstractly interpret values as members of the set $2 = \{0,1\}$, where being 0 means a cons cell definitely does not escape in the value, but 1 means a cons cell may escape in the value.

              Then, we can abstractly interpret $k$-argument, 1-result functions as being $2^k \rightarrow 2$, where the meaning is analogous to strictness analysis.

              We give the following abstract interpretations:

              $cond^{\#}(c,x,y) = \lambda c,x,y. x \vee y$

              $cons^{\#}(h, t) = \lambda h,t. t$

              $hd^{\#}(l) = 0$ (since we have no lists of lists).

              $tl^{\#}(l) = l$

              \item
                We solve these using fixed-point iteration.

                We first yield:

                $f^{\#}(x,y,z) = y \vee z$

                $g^{\#}(x,y) = g^{\#}(x,y)$

                $h^{\#}(x,y) = x \vee h^{\#}(x,y)$

                $k^{\#}(x,y) = y \vee k^{\#}(x,y)$

                So we have $f$ immediately, and then we initialise each $g^{\#}, h^{\#}, k^{\#}$ to $\lambda x,y. 0$, and substitute to get:

                $g^{\#}(x,y) = 0$

                $h^{\#}(x,y) = x$

                $k^{\#}(x,y) = y$

                $g^{\#}$ did not change so we are done, and then we iterate again for $h$ and $k$:

                $h^{\#}(x,y) = x$

                $k^{\#}(x,y) = y$

                Neither change, so we are done for those as well.

              
          \end{enumerate}



        
    \end{enumerate}
        
    \end{enumerate}
\end{document}
