\documentclass[12pt]{article}
\usepackage[margin=1.5cm]{geometry}
\usepackage{parskip}
\usepackage{amsmath}
\usepackage{amssymb}
\usepackage{amsfonts}
\usepackage{enumitem}
\usepackage{graphicx}
\usepackage{stmaryrd}
\graphicspath{ {./images/} }


\begin{document}
\begin{enumerate}[label=(\alph*)]
  \item
    \begin{enumerate}[label=(\roman*)]
      \item

        We construct the following two data-flow equations:

        $in-R(n) = \bigcup_{p \in pred(n)} out-R(p)$

        $out-R(n) = in-R(n) \setminus kill(n) \cup gen(n)$

        Where $kill(n)$ is the set of variables that $n$ kills (reads from), and $gen(n)$ is the set of variables that $n$ generates (writes to).

        We combine them into a single data-flow equation:

        $R(n) = \bigcup_{p \in pred(n)} R(p) \setminus kill(p) \cup gen(p)$

        Then, we provide the following algorithm which reports on such variables having anomalies. We initialise all the sets to the empty set to ensure we calculate the least solution.

\begin{verbatim}
R[1..n] = {}
while (R[] changes)
  for i in 1..n do
    R[i] = big_union(p in pred(n), R(p) \ kill(p) U gen(p))


for i in 1..n do
  for v in R[i] do
    print("Anomaly detected for variable", v)
\end{verbatim}

\item
  The issue with address-taken local variables is that we can now have implicit reads and writes to our variables. Naively, we could modify our algorithm to never give warnings for write-write anomalies for any address-taken local variables, or consider implicit reads to access all address-taken variables. A more sophisticated solution might do some points-to analysis to find what variables might point to others, and use this to account for implicit reads and writes.

  Similarly, global variables might be altered by procedures called during a function, so we could model all global variables as being read by any called function.
        
    \end{enumerate}

  \item
    \begin{enumerate}[label=(\roman*)]

      \item

\begin{verbatim}
a = rand()
b = rand()
print(a,b)

b = rand()
c = rand()
print(b,c)

c = rand()
d = rand()
print(c,d)

d = rand()
a = rand()
print(d,a)
\end{verbatim}

\item
  In a $k$-cyclic graph, we might be able to colour the graph in just 2 colours, by alternating the colour at every node (and we can clearly not do it in 1 colour). However, this requires that $k$ is even.

  If $k$ is odd, we can colour the graph in at minimum 3 colours, any strategy for colouring in 2 colours will always have two adjacent nodes with the same colour.

  Therefore, we give:

  $c(k) = \begin{cases}2 & k\text{ is even}\\3 & \text{otherwise}\end{cases}$

        
    \end{enumerate}
        
\end{enumerate}
\end{document}
