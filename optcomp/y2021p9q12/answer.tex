\documentclass[12pt]{article}
\usepackage[margin=1.5cm]{geometry}
\usepackage{parskip}
\usepackage{amsmath}
\usepackage{amssymb}
\usepackage{amsfonts}
\usepackage{enumitem}
\usepackage{graphicx}
\usepackage{stmaryrd}
\graphicspath{ {./images/} }


\begin{document}
\begin{enumerate}[label=(\alph*)]

  \item
    LICM is a transformation that detects assignments in loops that are invariant: i.e. those whose expressions are the same on every iteration of the loop. Then, it moves these assignments to just before entry to the loop.

    In the code above, we should move the assignment \texttt{c = b + 5} to just before entry to the loop, since the value of \texttt{b} cannot change throughout the loop.

    \item
      The data-flow analysis we propose is to find those expressions who only use variables that are redefined within the loop body.

      To do this, our analysis should produce, for each line of code, a set of definitions that can reach that line:

      $in-rd(n) = \bigcup_{p \in pred(n)} out-rd(p)$

      $out-rd(n) = in-rd(n) \setminus kill(n) \cup gen(n)$

      Where $kill(n)$ is the set of definitions that assign to the same variable that $n$ assigns to, and $gen(n)$ is the singleton set of the definition in $n$, if there is one.

      Then, we combine:

      $rd(n) = (\bigcup_{p \in pred(n)} rd(p) \setminus kill(p) \cup gen(p))$

      To deal with address-taken variables, we consider that any indirection can access any address-taken variable.

\begin{verbatim}
l0: a = read();           {}
l1: b = read();           {l0}
l2: p = &a;               {l0,l1}
l3: q = &b;               {l0,l1,l2}
l4: r = &p;               {l0,l1,l2,l3}}
l5: if (read() > 0) {     {l0,l1,l2,l3,l4}
l6:   a = b + 5;            {l1,l2,l3,l4}
l7: } else {
l8:   i = 0;                {l0,l1,l2,l3,l4}
l9:   while (i < 10) {      {l0,l1,l2,l3,l4,l8,l10,l11,l12}
l10:     c = b + 5;         {l0,l1,l2,l3,l4,l8,l10,l11,l12}
l11:     **r += *q;         {l0,l1,l2,l3,l4,l8,l10,l11,l12}
l12:     i += 1;            {l3,l4,l8,l10,l11,l12}
l13:   }
l14:   a += c;              {l0,l1,l2,l3,l4,l8,l10,l11,l12}}
l15: }
l16: print(a);              {l1,l2,l3,l4,l6,l8,l10,l11,l12,l14}

\end{verbatim}


\item
   Here, we do not see that \texttt{l10} is invariant, because the definition at \texttt{l11} reaches it, and we make the conservative assumption that all indirect writes must access all address-taken variables, so the write to \texttt{**r} might access \texttt{b}.

   \item
     An analysis that might make this more accurate is Andersen's analysis, which is a flow insensitive analysis that attempts to find what variables other variables might point to. It uses an inference system:

     \begin{align*}
       \frac{}{\vdash x := \&y : y \in pt(x)}\\\\
       \frac{}{\vdash x := y : pt(x) \supseteq pt(y)}\\\\
       \frac{u \in pt(x) \hspace{15pt}v \in pt(u) \hspace{15pt} w \in pt(y)}{\vdash **x := *y : pt(v) \supseteq pt(w)}
     \end{align*}

     \item
       After running the analysis, we get the following results:

       $pt(p) = \{a\}$
       
       $pt(q) = \{b\}$

       $pt(r) = \{p\}$

       $pt(b) = \{\}$

       $pt(a) = \{\}$

       $pt(i) = \{\}$

       With these results, by running LICM again, we get:

\begin{verbatim}
l0: a = read();           {}
l1: b = read();           {l0}
l2: p = &a;               {l0,l1}
l3: q = &b;               {l0,l1,l2}
l4: r = &p;               {l0,l1,l2,l3}}
l5: if (read() > 0) {     {l0,l1,l2,l3,l4}
l6:   a = b + 5;            {l1,l2,l3,l4}
l7: } else {
l8:   i = 0;                {l0,l1,l2,l3,l4}
l9:   while (i < 10) {      {l0,l1,l2,l3,l4,l8,l10,l11,l12}
l10:     c = b + 5;         {l0,l1,l2,l3,l4,l8,l10,l11,l12}
l11:     a += b;            {l0,l1,l2,l3,l4,l8,l10,l11,l12}
l12:     i += 1;            {l1,l2,l3,l4,l8,l10,l11,l12}
l13:   }
l14:   a += c;              {l0,l1,l2,l3,l4,l8,l10,l11,l12}}
l15: }
l16: print(a);              {l1,l2,l3,l4,l6,l8,l10,l11,l12,l14}
\end{verbatim}

Noe, we see that at \texttt{l10}, the only definition of \texttt{b} that reaches it is at \texttt{l1}, which is not inside the loop, so we can safely hoist it.

\item

  Another optimisation that could be applied is code hoisting. To perform this transformation, we use very-busy-expression analysis (which finds all expressions that must be calculated in every future path). If we find such an expression, we can lift it.

  In this case, after we left \texttt{c = b + 5}, we get that \texttt{b+5} is busy just before \texttt{l4}, so we get:

\begin{verbatim}
l0: a = read();          
l1: b = read();         
l2: p = &a;            
l3: q = &b;          
l4: r = &p;         
l5: t = b + 5;
l6: if (read() > 0) { 
l7:   a = t;     
l8: } else {
l9:   i = 0;        
l10:  c = t;   
l11:   while (i < 10) { 
l12:     a += b;     
l13:     i += 1;    
l14:   }
l15:   a += c;     
l16: }
l17: print(a);    
\end{verbatim}





        
    \end{enumerate}
\end{document}
