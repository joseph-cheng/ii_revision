\documentclass[12pt]{article}
\usepackage[margin=1.5cm]{geometry}
\usepackage{parskip}
\usepackage{amsmath}
\usepackage{amssymb}
\usepackage{amsfonts}
\usepackage{enumitem}
\usepackage{graphicx}
\usepackage{stmaryrd}
\graphicspath{ {./images/} }

\begin{document}
\begin{enumerate}[label=(\alph*)]
    \item
        Available expression analysis is a forwards data-flow analysis to keep track of all expressions who have had their values calculated (and not subsequently invalidated by a variable re-definition).

        It works using the following data-flow equations:

        \[
            in-avail(n) = \bigcap_{p \in pred(n)} out-avail(p)
        .\] 

        \[
            out-avail(n) = in-avail(n) \setminus kill(n) \cup gen(n)
        .\] 

        Where $kill(n)$ is the set of expressions that $n$ changes a variable in, and does not subsequently recompute the value of, and $gen(n)$ is the set of expressions that are always calculated by $n$ and that $n$ does not subsequently redefine any variables in

        This leads to the holistic equation:

        \[
            avail(n) = \begin{cases}\bigcap_{p \in pred(n)} (avail(p) \setminus kill(p) \cup gen(p)) & pred(n) \neq \emptyset \\ \{\} & pred(n) = \emptyset\end{cases}
        .\] 

        (We require the second case so we do not mistakenly assume every expression is available at the entrypoint to a program)

        This makes sense because an expression is only available if it is calculated in every possible path up to a node. We safely underestimate this with syntactic availability, since it is undecidable in general to determine the \textit{possible} control flow paths in a program.

        An algorithm works as follows, since we are able to enumerate all expressions in a program (and provide a bitmask for their availability).

        Then, we construct an array \texttt{avail}, which contains the set of available expressions at node \texttt{i}:

\begin{verbatim}
avail[0] = {}
while (avail[] changes) {
  for (int ii=0; ii < n; ii++) {
    avail[i] = big_intersect(p <- pred(i), avail(p) \ kill(p) U gen(p))
  }
}
\end{verbatim}

\item
    Available expression analysis helps to perform common sub-expression elimination by finding sub-expressions that are available at a node, and replacing their calculation with a MOV from an already existing calculation of the expression.

    More precisely, we find a node $n$ that calculates an expression $e$, such that $e \in avail(n)$. Then, we replace the calculation of $e$ with an instruction that copies a register $t$ into the destination register instead. Then, we backtrack on every path from $n$, finding the first instance where $e$ is calculated, make that instruction calculate into $t$ instead, then insert a copy from $t$ into the original destination register.

    We then repeat this until we cannot find any more expressions in nodes to eliminate.

\item
    The following code might get naively compiled down to the following:

\begin{verbatim}
t1 = b * c
arg1 = a + t1
t2 = b * c
arg2 = a + t2
u = call f
t3 = b * c
t4 = a + t3
v[t4] = u
\end{verbatim}

The described algorithm for CSE will eliminate the second calculation of \texttt{b*c}, with something like the following:

\begin{verbatim}
s0 = b * c
t1 = s0
arg1 = a + t1
t2 = s0
arg2 = a + t2
u = call f
t3 = s0
t4 = a + t3
v[t4] = u
\end{verbatim}

But we are no longer able to perform any more common sub-expression elimination, even though there is some (since \texttt{t1} and \texttt{t2} will always be the same). Since \texttt{b} and \texttt{c} are local, we can eliminate the calculation after the call to \texttt{f}.

\item
    Even though neither \texttt{b} nor \texttt{c} are modified in our loop, \texttt{b * c} is not available in our do-while loop, since \texttt{b*c} is not available on first entry to our loop.

    In the second example, however, \texttt{b*c} is available since it is calculated both on first entry, and in subsequent iterations of the loop.

    This means that we can perform CSE to replace \texttt{x} with \texttt{x += t} and \texttt{z = b * c} with \texttt{t = b*c; z = t}

    This has effectively lifted a loop-invariant out, but required that we specifically pre-calculate the expression \textit{before} the loop begins, so this will not work for every loop-invariant, and a more sophisticated loop-invariant lifting analysis/algorithm would be able to find more instances of this.


        
    \end{enumerate}
\end{document}
