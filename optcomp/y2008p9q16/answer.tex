\documentclass[12pt]{article}
\usepackage[margin=1.5cm]{geometry}
\usepackage{parskip}
\usepackage{amsmath}
\usepackage{amssymb}
\usepackage{amsfonts}
\usepackage{enumitem}
\usepackage{graphicx}
\usepackage{stmaryrd}
\graphicspath{ {./images/} }

\usepackage{stmaryrd}

\begin{document}
\begin{enumerate}[label=(\alph*)]
  \item
    We define the effect system as follows, where we change typing judgments from $\Gamma \vdash e : T$ to $\Gamma \vdash e : T, F$, where $F$ is a set of effects. We also alter function types to be $T \xrightarrow{F} T'$, to represent the effects that occur upon application of the function (latent effects). We assume the only other type is integers, and this is what channels read/write.

    \begin{align*}
      \frac{}{\Gamma \vdash x : T, \{\}}(\Gamma(x) = T)\\
      \frac{\Gamma, x : T \vdash e : T', F}{\Gamma \vdash \lambda x. e : T \xrightarrow{F}T', \{\}}\\
      \frac{\Gamma \vdash e_1 : T \xrightarrow{F} T', F' \hspace{15pt} \Gamma \vdash e_2 : T, F''}{\Gamma \vdash e_1 e_2 : T', F \cup F' \cup F''}\\
      \frac{\Gamma \vdash e_1 : T, F \hspace{15pt} \Gamma \vdash e_2 : T', F'}{\Gamma \vdash e_1;e_2 : T', F \cup F'}\\
      \frac{\Gamma, x : int \vdash e : T, F}{\Gamma \vdash \xi?x.e : T, F \cup \{R_\xi\}}\\
      \frac{\Gamma \vdash e_1 : int, F \hspace{15pt} \Gamma \vdash e_2 : T, F'}{\Gamma \vdash \xi!e_1.e_2 : T, F \cup F' \cup \{W_\xi\}}\\
      \frac{\Gamma \vdash e_1 : int, F \hspace{15pt} \Gamma \vdash e_2 : T, F' \hspace{15pt} \Gamma \vdash e_3 : T, F''}{\Gamma \vdash \texttt{if }e_1\texttt{ then }e_2\texttt{ else }e_3 : T, F \cup F' \cup F''}
    \end{align*}

    The two principle occurrences of effects in the system are immediate effects, which occur in the effect judgement, and latent affects (as described earlier) which occur in the typing judgement.

  \item
    If we consider $\llbracket e \rrbracket = (v,f)$, where $v$ is the value that $e$ evaluates to, and $f$ is the set of effects that occur during execution of $e$, then safety can be stated as follows:

    If $\{\} \vdash e : T, F$, then $v \in \llbracket t\rrbracket \wedge f \subseteq F$, where $(v,f) = \llbracket e \rrbracket$, i.e. that any well-typed expression $e$ evaluates to a value of its typing judgment, and the actual effects are a subset of the predicted effects.

  \item
    This effect system is any-path like (although the typing system is not). If $e_1, e_2, e_3$ have effects $F, F', F''$, then it is clear to see that the analysis of the conditional will yield effects $F \cup F' \cup F''$ immediately.

    Then, analysis of $(e_1;e_2)$ gives effects $F \cup F'$, so analysis of $(e_1;e_2);e_3 = e_1;e_2;e_3$ has effects $F \cup F' \cup F''$.

  \item

    We provide the following two new typing/effect rules:

    \[
      \frac{\Gamma \vdash e : T, F}{\Gamma \vdash \texttt{letchan }\xi\texttt{ in }e : T', F \setminus \{R_\xi, W_\xi\}}(\text{$T' = T$ but with latent effects referencing $\xi$ removed})
    .\] 

    \[
      \frac{\Gamma \vdash e_1 : int, F \hspace{15pt} \Gamma \vdash e_2 : int, F'}{\Gamma \vdash \texttt{parsum}(e_1, e_2) : int, F \cup F'}
    .\] 

  \item
    Here, a set of lists/arrays would be suitable. Each list would contain a possible sequence of effects, and then $F$ would be a set of all the lists, so what possible sequences of effects we can have.

    It should be noted that this might end up generating lots of lists of effects for things like the \texttt{cond} construct, since we need to generate a separate list for each branch.

    In the example, $F$ might look like $\{[R_{\xi_1}, R_{\xi_2}, W_{\zeta}], [R_{\xi_2}, W_\zeta]\}$




        
\end{enumerate}
\end{document}
