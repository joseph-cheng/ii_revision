\documentclass[12pt]{article}
\usepackage[margin=1.5cm]{geometry}
\usepackage{parskip}
\usepackage{amsmath}
\usepackage{amssymb}
\usepackage{amsfonts}
\usepackage{enumitem}
\usepackage{graphicx}
\usepackage{stmaryrd}
\graphicspath{ {./images/} }


\begin{document}
\begin{enumerate}[label=(\alph*)]

  \item
    \begin{enumerate}[label=(\roman*)]

      \item
        We define three abstract values, 0, 1, or 2, representing not a tree, definitely a tree, and maybe a tree respectively.

        0 represents concrete values of \texttt{graph\_node}s that have cycles (i.e. are not trees). 1 represents concrete values of \texttt{graph\_node}s that do not have cycles (i.e. are trees), and 2 represents concrete values of all \texttt{graph\_node}s.

        This is safe because if graph node is a tree, then it will either be 0 or 2, and if it is not a tree then it will be either 1 or 2, so properties exhibited by the abstract interpretation hold in the concrete world.

        \item
          \begin{enumerate}[label=(\Alph*)]

            \item

              We provide the following `matrix':

\begin{verbatim}
g:   0 1 2
ret: 1 1 1
\end{verbatim}

\item
  We provide the following matrix:

\begin{verbatim}
 g 0 1 2
c  
0  0 0 0
1  0 2 2
2  0 2 2
\end{verbatim}

Even adding a tree as a child to another tree does not always result in a tree, for example we could call \texttt{add\_child(g,g)}, and the result is clearly not a tree.

\item
  We provide the following matrix:

\begin{verbatim}
 g 0 1 2
c
0  2 1 2
1  0 1 2
2  2 1 2
\end{verbatim}

\item
   We provide the following matrix:

\begin{verbatim}
g   0 1 2
ret 2 1 2
\end{verbatim}
          \end{enumerate}


        
    \end{enumerate}

          \item

            We consider the following tuple as an abstract value $(n, b)$ where $n$ tells us the shortest path to a leaf node, and $b$ is a Boolean that is true if $n$ is exactly correct, or false if the actual value is greater than or equal to $n$.

            Functions:

\begin{verbatim}
create\_child(g):

(n, b) -> (0, T)


add\_child(g,c)

(n, T) x (m, F) -> (min(n, m+1), T if n <= m+1 else F)
(n, T) x (m, T) -> (min(n, m+1), T)
(n, F) x (m, T) -> (min(n, m+1), T if m+1 <= n else F)
(n, F) x (m, F) -> (min(n, m+1), F)


remove_child(g,c)

(n, b1) x (m, b2) -> (n, b) if n < m+1
(n, b1) x (m, b2) -> (n, F) if n >= m+1


dfs(g,v)
(0, T) -> (0, T)
(n, b) -> (0, F)
\end{verbatim}

      
  \end{enumerate}
\end{document}
