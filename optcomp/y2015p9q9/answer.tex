\documentclass[12pt]{article}
\usepackage[margin=1.5cm]{geometry}
\usepackage{parskip}
\usepackage{amsmath}
\usepackage{amssymb}
\usepackage{amsfonts}
\usepackage{enumitem}
\usepackage{graphicx}
\usepackage{stmaryrd}
\graphicspath{ {./images/} }


\begin{document}
\begin{enumerate}[label=(\alph*)]
  \item

    A classical data-flow analysis can be seen as the format described by emitting a constraint at each node based on the definition of our $live$ set (or equivalent set for other analyses).

    For example, at a node $n$, we simply emit the constraint $live(n) = \bigcup_{s \in succ(n)} live(s) \setminus kill(n) \cup gen(n)$, where $kill(n)$ is the set of variables that a node kills, i.e. it defines, and $gen(n)$ is the set of variables that a node references.

    Since we can enumerate every variable used by a program, we can turn our sets of variables into bit-arrays, one element for each variable, which is the space of values that equality will act on.

    To be a `least' solution, we mean that there is no other solution such that any live sets for any node are smaller (wrt $\subseteq$) then our least solution.

  \item
    To do this, we simply generate the inequality constraints $live(n) \supseteq \bigcup_{s \in succ(n)} live(s) \setminus kill(n) \cup gen(n)$. It is important not to use $\subseteq$ here, or else the empty set is trivially a solution for each constraint. We also retain the same least solution as in (a), since the only additional solutions we allow are `bigger' than the set of solutions we allow for (a), so we cannot generate a new least solution.

  \item
    We can alter the flowgraph walker to instead emit a new constraint for each $s \in succ(n)$, i.e. if node $n$ has successors $s_1$ and $s_2$, then we generate two constraints: $live(n) \supseteq live(s_1) \setminus kill(n) \cup gen(n)$ and $live(n) \supseteq live(s_2) \setminus kill(n) \cup gen(n)$.

    By moving to $\supseteq$ instead of $=$, we allow the creation of these alternative constraints.

    In the example, we treat each line of code as a separate node, numbered 1 through 5

    It generates the following constraints:

    $L_1 \supseteq L_2 \setminus \{x\} \cup \{\}$

    $L_2 \supseteq L_3 \setminus \{y\} \cup \{\}$

    $L_3 \supseteq L_4 \setminus \{z\} \cup \{y\}$

    $L_4 \supseteq L_5 \setminus \{\} \cup \{x\}$

    $L_4 \supseteq L_6 \setminus \{\} \cup \{x\}$

    $L_5 \supseteq L_6 \setminus \{\} \cup \{z\}$

  \item

    The purpose of 0-CFA is to try and compute the possible values that are available at any given `program point'. This is useful in trying to determine possible control flows. Naively, when we perform a jump to some value not known at compile-time, our flow-graph might include edges to every instruction for safety. 0-CFA lets us potentially reduce the amount of edges we have to generate, since a program point cannot branch to a value if there was no way for it to flow to it.

    It fits the description of `walking a data structure and emitting constraints' relatively well, but the data structure in this case is a syntax tree of our program, as opposed to a flow graph. At each node in the tree, we can emit a constraint.

  \item
    For this example, we will take the case of function calls in a functional programming language. Since functions are first-class, we do not necessarily know the destination of a function when we apply it.

    This means that we cannot immediately generate 0-CFA constraints, since this would require knowing the possible function values at time of constraint generation, which is the whole point of the analysis in the first place!

    For this reason, we generate meta-constraints. For example, suppose we have expression $(\_^a \_^b)^c$, where each $^i$ represents program point $i$. For this, we generate meta-constraints $(\lambda x^p. e^q) \in \alpha_a \implies \alpha_p \supseteq \alpha_b$ and $(\lambda x^p. e^q) \in \alpha_a \implies \alpha_c \supseteq \alpha_q$, where each $\alpha_i$ represents the set of flow-values at program point $i$. We generate these constraints because the argument to out function (at program point $b$) can flow to the formal parameter in any function that program point $a$ might have, and the return value of that function can flow to the overall value of the function application (at program point $c$).
    \end{enumerate}
\end{document}
