\documentclass[12pt]{article}
\usepackage[margin=1.5cm]{geometry}
\usepackage{parskip}
\usepackage{amsmath}
\usepackage{amssymb}
\usepackage{amsfonts}
\usepackage{enumitem}
\usepackage{graphicx}
\usepackage{stmaryrd}
\graphicspath{ {./images/} }


\begin{document}
\begin{enumerate}[label=(\alph*)]

  \item
    \begin{enumerate}[label=(\roman*)]

      \item
        \begin{enumerate}[label=(\Alph*)]

          \item

            To decrypt with ECB, we simply read up on the table:

            \texttt{0ddba11}

          \item
            To decrypt with CBC, we decrypt each block using the table, and then XOR with the previous ciphertext block to get the plaintext block:

            \texttt{cafe}

          \item
            To decrypt with OFB, we first construct out $R$ chain, by repeatedly applying $E_K$

            \texttt{R = e50cb}

            Then, we simply XOR this with our ciphertext:

            \texttt{feed}

            
        \end{enumerate}

      \item
        To calculate the CBC-MAC, we encrypt with CBC (with IV 0), and take the final block as our MAC:

        Encrypting, we get:

        \texttt{3e2b}, so our final MAC is \texttt{b}.

      \item

        We know that the characters \texttt{fc} correspond to \texttt{e1} in our message.

        So, we know that $f = R_5 \oplus e$

        And, that $c = R_6 \oplus 1$

        Therefore, we know that $f \oplus f = R_5 \oplus e \oplus f$, or that $0 = R_5 \oplus 1$

        Similarly, we know that $c \oplus 1 = R_6 \oplus 1 \oplus 1$, or that $d = R_6 \oplus 0$

        Therefore, we should replace \texttt{fc} with \texttt{0d}.

        Finally, we know from the check digit is some value $n$ such that $n = m_1 \oplus m_2 \oplus m_3 \oplus m_4 \oplus e \oplus 1$

        So, we know that $n \oplus e = m_1 \oplus m_@ \oplus m_3 \oplus m_4 \oplus 1 \oplus 0$, forming a valid check digit for our false message.

        Furthermore, we know that $R_7 \oplus n = 2$, so $R_7 \oplus n \oplus e = c$

        Therefore, our encrypted check digit should be \texttt{c}.

        So, we get the final message:

        \texttt{a59de0dc}
    \end{enumerate}

  \item
    Suppose we had only 8192 instances of the sub-block \texttt{0000}. After encryption, we reduce the possible values of the permutation, since each location of the sub-block \texttt{0000} can only map to one of 8192 other places. Essentially, we can divide the possible destinations for each location by $2^4 = 16$ for each block, so we require $\lceil \log_{16} (\frac{32768 \cdot 8}{4} ) \rceil = 4$ blocks.

    We could construct blocks as follows:

    First, number each of the sub-block positions. There are $2^16$ sub-block positions, so this is a 4-digit hexadecimal number.

    For the first block, give each sub-block the value of the most significant digit in the sub-block position number.

    For the second block, give each sub-block the value of the next most significant digit in the sub-block position number.

    For the third block, give each sub-block the value of the next most significant digit in the sub-block position number.

    For the fourth block, give each sub-block the value of the next most significant digit in the sub-block position number.

    At this point, for example, if we wanted to find the destination of the first sub-block, we look at all the occurrences of \texttt{0}s in the resulting sub-blocks and take their intersection, which should leave us with 1 block.






        
    \end{enumerate}
\end{document}
