\documentclass[12pt]{article}
\usepackage[margin=1.5cm]{geometry}
\usepackage{parskip}
\usepackage{amsmath}
\usepackage{amssymb}
\usepackage{amsfonts}
\usepackage{enumitem}
\usepackage{graphicx}
\usepackage{stmaryrd}
\graphicspath{ {./images/} }


\begin{document}
\begin{enumerate}[label=(\alph*)]
  \item
    We give the three properties below:

    \begin{itemize}
      \item
        \textbf{Pre-image resistance}

        This requires that given an output value of a $h$, say $y$, no PPT adversary exists that can find an $x$ such that $h_s(x) = y$ with non-negligible probability.

      \item
        \textbf{Second pre-image resistance}

        This requires that given an input value of $h$, say $x$, no PPT adversary exists that can find an $x'$ such that $h_s(x) = h_s(x')$ with non-negligible probability.

      \item
        \textbf{Collision resistance}

        This requires that no PPT adversary exists that can find a pair $x,x'$ such that $h_s(x) = h_s(x')$ with non-negligible probability.
    \end{itemize}

  \item
    We can implement a one-time signature scheme by generating 256 uniformly randomly generated 128-bit strings, call them $R_{i,j}$, where $i$ is either 0 or 1, and $1 \leq j \leq 128$. This is our secret key. We then distribute our public key as the hash of each $R_{i,j}$

    Then, we can generate a hash of our message $h(M)$ and use each hash bit to choose an $i$ in $R_{i,j}$.

    More formally, suppose $h(M) = b_1b_2\ldots b_n$

    Then our signature becomes $(R_{b_1,1}, R_{b_2,2}, \ldots,  R_{b_n,n})$

    To verify this signature, the recipient can apply $h$ to each string in the signature to create $(h(R_{b_1,1}), \ldots, h(R_{b_n, n}))$, and they can also calculate $h(M)$ and use the public key to verify whether this matches the expected $h(R_{b_i, i})$.

    Intuitively, an adversary should not be able to forge a signature because they cannot know what any of the $R_{i,j}$ are (assuming one-time usage).

  \item
    We require that $K$ is a permutation, i.e. a bijection. If it is not, then there are at least two elements in the range of $K$ that correspond to different elements in the domain, at which point we can lose information.

  \item
    Not relevant

  \item
    Since each character can be one of any 32 values, by using the correct messages, we can divide down the set of possible inputs in $\{0,\ldots,999\}$ for any output of $K$ by 32. This means that we requires $\log_{32}(1000) < 2$, so 2 blocks, to recover $K$.

    For example, in the first block, we could arrange the content to have the first 32 characters be $a_0$, the next 32 by $a_1$, and the last 8 be $a_{31}$ (since we do not have exactly 1024 characters in a message, but this is fine).

    Upon receiving the permutation of this block, we then know that, for example, $K(0)$ can be one of 32 values (determined by the locations of the $a_0$s), and the same for all other input values (except for $K(992)$ to $K(999)$, of which we know 8 possible values).

    Then, our second block we form by just repeating $a_0$ to $a_31$ in sequence until the end of our message.

    Upon receiving the permutation of this block, we can, for example, look at the locations of the $a_0$s. Only one of these locations will be one of the possible values for $K(0)$ that we deduced from the previous block, so we can then deduce the value for $K(0)$. Then, only one of these locations will be one of the possible values for $K(32)$ that we deduced from the previous block, so we can deduce the value $K(32)$, etc. We can then repeat this for each $a_i$ and each block of 32, which lets us recover the entirety of $K$.



        
\end{enumerate}
\end{document}
