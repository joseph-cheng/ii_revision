\documentclass[12pt]{article}
\usepackage[margin=1.5cm]{geometry}
\usepackage{parskip}
\usepackage{amsmath}
\usepackage{amssymb}
\usepackage{amsfonts}
\usepackage{enumitem}
\usepackage{graphicx}
\usepackage{stmaryrd}
\graphicspath{ {./images/} }


\begin{document}
\begin{enumerate}[label=(\alph*)]

  \item
    \begin{enumerate}[label=(\roman*)]

      \item
        The \textbf{Data} Encryption Standard defines a 16-round Feistel cipher.

      \item
        Files encrypted with Cipher Block Chaining start with a \textbf{random} initial vector.

      \item
        Not relevant.

      \item
        Not relevant.

      \item
        Not relevant.
        
    \end{enumerate}

  \item
    Not relevant.

  \item
    Double DES feeds a plaintext $P$ into an instance of DES using key $K_1$ to create $C_1$, and then feeds $C_1$ into an instance of DES using key $K_2$ to create $C_2$, the output.

    If we have an arbitrary fixed $P$ that encrypts to known $C$, then by encrypting $P$ once we can try all $2^{56}$ possible keys to build a table mapping keys for the first instance of DES to the intermediate value $C_1$.

    Then, by decrypting $C$ once, we can try all $2^{56}$ possible keys to build a table mapping keys for the second instance of DES to the intermediate value $C_1$.

    Then, we simply have to find two values of $C_1$ in both tables that match, which gives us the keys for the two instances of DES.

    This only required time and space of order $2 \cdot 2^{56} = 2^{57}$.

    Triple DES requires that we perform the meet-in-the-middle either after the first instance of DES (which requires a $2^{112}$ table to be built from the decryption side), or after the second instance of DES (which requires a $2^{112}$ table to be built from the encryption side), which is less feasible.

  \item
    The Vignere cipher is unconditionally secure if the key is as long as, or longer, than the plaintext.

    If the key is shorter than the plaintext, then an adversary could try all possible keys shorter than the plaintext, and identify the correct plaintext from the other.

    We also require that the key is picked uniformly randomly, used only once, and not revealed to the adversary.
        
\end{enumerate}
\end{document}
