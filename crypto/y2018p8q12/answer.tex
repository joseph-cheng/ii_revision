\documentclass[12pt]{article}
\usepackage[margin=1.5cm]{geometry}
\usepackage{parskip}
\usepackage{amsmath}
\usepackage{amssymb}
\usepackage{amsfonts}
\usepackage{enumitem}
\usepackage{graphicx}
\usepackage{stmaryrd}
\graphicspath{ {./images/} }


\begin{document}
\begin{enumerate}[label=(\alph*)]
  \item
    We give three types of group:

    \begin{itemize}
      \item
        \textbf{Schnorr groups}

        Schnorr groups are subgroups of $\mathbb{Z}^*_p$ that have prime-order, where the group operator is multiplication. Since these have prime order, any group element is a generator. Furthermore, the prime order means that it has no non-trivial subgroups, so there are no subgroup confinement attacks.

        The set of elements for a Schnorr group can be generated by picking $p,q,r$ with $p,q$ prime such that $p = qr + 1$. Then, find an $h$ smaller than $p$ such that $h^r \mod{p} \neq 1$, and finally using $h^r \mod{p}$ as a generator.

      \item
        \textbf{Elliptic curves over a prime ordered group $\mathbb{Z}_p$ with $p > 3$}

        These groups are parameterised by two numbers $a,b$ to generate the curve $y^2 = x^3 + ax + b$ (due to the condition that $p > 3$)

        We define group elements as being points on the elliptic curve with their elements taken from $\mathbb{Z}_p$, as well as a point at infinity $\mathcal{O}$, that is the neutral element. The group operator comes geometrically: drawing a line between the two operands, and taking the resulting third intersection point with the curve as the result.

      \item
        \textbf{Elliptic curves over $GF(2^n)$}

        These groups are parametrised by $a,b,c$ to generate the curve $y^2 + cy = x^3 + ax + b$, which is the resulting simplified Weierstrass equation if the field has characteristic $2$.

        We have the same group elements and group operator as for the previous elliptic curve groups, but the main benefit is that the computation required for computing the group operator is simplified, due to the different elliptic curve equation.


    \end{itemize}

  \item
    \begin{enumerate}[label=(\roman*)]

      \item
        This would take $\frac{64^8}{10^6} \approx 280 \cdot 10^6$ seconds

      \item
        This would take $\frac{64^8}{10^9} \approx 280000$ seconds

      \item
        When building a rainbow table, the GPU cluster will also have to evaluate the reduction functions as often as MD5, as the reduction functions are used to generate the targets for computing MD5, by turning MD5 hashes back into a password.

      \item
        If we execute MD5 $2^{50}$ times whilst generating the rainbow table, and we have $2^{32}$ key-value pairs, then each chain `contains' $2^{18}$ elements.

        Since, in the worst case, we will be at the worst end of the chain, this could take up to $(2^{18})(2^{18} - 1) \cdot \frac{1}{2}$ steps, which is around $2^{35}$.

        On the laptop, this takes $\frac{2^{35}}{10^6} \approx 34000$ seconds.
        
    \end{enumerate}
        
\end{enumerate}
\end{document}
