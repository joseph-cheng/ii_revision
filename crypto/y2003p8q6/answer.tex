\documentclass[12pt]{article}
\usepackage[margin=1.5cm]{geometry}
\usepackage{parskip}
\usepackage{amsmath}
\usepackage{amssymb}
\usepackage{amsfonts}
\usepackage{enumitem}
\usepackage{graphicx}
\usepackage{stmaryrd}
\graphicspath{ {./images/} }


\begin{document}
\begin{enumerate}[label=(\alph*)]

  \item
    Here, $g$ is the generator of some group. $q$ is the order of that group. Since we take $g^k \mod p$, it appears that the group is a subgroup of $Z^*_p$

    $k$ is some randomly chosen secret key generated per signature, and $x$ is a secret key that is held for a longer period of time (i.e. reused). $M$ is the message the signature is being taken of, and $h$ is a secure hash function.

    Finally, $r$, $s$ are the two components of the signature.

    \item
      To verify a signature, consider what happens if $r,s$ are generated as expected, where $g^k$ here means applying the group operator to $g$ $k$ times, not anything to do with integer exponentiation.

      $g^{ks + xr} = g^{h(M)}$
        
      $g^{ks} = g^{h(M)}(g^{x})^{-r}$

      $g^k = g^{h(M)s^{-1}}(g^x)^{-rs^{-1}}$

      Since a verifier knows $g^x$ (the public key), $r$, $s$, $M$, and $h$, they can compute the RHS, and check whether it equals $r$ (which is $g^k$).

      \item
        If an implementation does not check this, then for example an adversary could generate a signature $(r,s)$ with $r=0$, choosing an arbitrary $k$ and thus $s = \frac{h(M)}{k} \pmod{q}$, which would be a valid signature generated without needing $x$.

        Choosing and $r$ greater than or equal to $q$ should not have an impact since it will be reduced by the modulus (unless $r$ is a multiple of $q$, when it is the same as if it is zero).

        \item
          The consequences of this are that we can forge a signature for any message.

          Under the proposed scheme, the verifier would check that:

          $r = g^{Ms^{-1}}y^{-rs^{-1}}$

          Now, suppose we choose $r = g^ay^b$, and $m = \frac{ar}{b}$, where $a,b$ are arbitrary.

          Then, let $s = \frac{r}{b}$

          We see that $g^{Ms^{-1}}y^{-rs^{-1}} = g^{a}y^{b}$.

          Therefore, we have a forgery.


    \end{enumerate}
\end{document}
