\documentclass[12pt]{article}
\usepackage[margin=1.5cm]{geometry}
\usepackage{parskip}
\usepackage{amsmath}
\usepackage{amssymb}
\usepackage{amsfonts}
\usepackage{enumitem}
\usepackage{graphicx}
\usepackage{stmaryrd}
\graphicspath{ {./images/} }


\begin{document}
\begin{enumerate}[label=(\alph*)]

  \item
    A pseudorandom function is a function that tries to imitate a random function, that is a randomly selected function from the set of all functions.

    More precisely, a function $F : \{0,1\}^k \times \{0,1\}^m \rightarrow \{0,1\}^n$ is pseudorandom if no distinguisher exists that can distinguish between $F_K$ (where $K$ is unknown to the distinguisher) and a function $f$, where $f$ is randomly chosen from the set of all functions $\{0,1\}^m \rightarrow \{0,1\}^n$ with non-negligible probability with respect to $k$.

  \item
    \begin{enumerate}[label=(\roman*)]

      \item
        In digital certificate generation, we sign a combination of an entity, their public key, a start time and a duration, and a serial number. To create this signature, we hash all of this information together to use a fixed length input.


        We require pre-image resistance, or else an attacker can create fake information that matches any certificate. 

        However, we do not require collision resistance, since the CA is trusted, and thus we can trust them not to create two certificates that hash to the same value, with the intent to substitute one certificate for the other later.

      \item
        These electronic signatures will be created by applying some signature to the contents of a contract (and perhaps some metadata). Again, this signature will be created by first applying a hash to the data we are signing, in order to use a fixed-length input.

        If the hash function is not collision resistant, then one of the companies could produce a second, fake contract, that hashes to the same value as the real contract, but the signature would verify this fake contract. Therefore, the hash function must be collision resistant (and thus pre-image resistant).

      \item
        Fingerprints on `known good' files can be generated by simply hashing the contents of the file.


        If the hash function is not collision resistant, then a software developer could submit a `known good' file to an anti-virus company, along with a virus that matches that hash, and thus create a virus that is recognised as a `known good' file.

      \item
        In this case, hash functions are used to generate elements on a block chain, where each element contains a transaction, and a hash of the previous transaction.

        If the hash function is not collision resistant, then an attacker could perform a legitimate transaction with another party, and create a malicious transaction that resulted in the same hash, and then broadcast a new hash chain but substituting the legitimate transaction for the malicious transaction, making false transactions seem legitimate on the hash chain.
        
    \end{enumerate}
        
    \end{enumerate}
\end{document}
