\documentclass[12pt]{article}
\usepackage[margin=1.5cm]{geometry}
\usepackage{parskip}
\usepackage{amsmath}
\usepackage{amssymb}
\usepackage{amsfonts}
\usepackage{enumitem}
\usepackage{graphicx}
\usepackage{stmaryrd}
\graphicspath{ {./images/} }


\begin{document}
\begin{enumerate}[label=(\alph*)]

  \item
    \begin{enumerate}[label=(\roman*)]

      \item

        Since digital signatures are a public-key concept, MACs use a shared private key. This means that if Alice provides a digital signature to Bob, this signature can also be verified by anyone else, and Alice might not want anyone else but Bob to know that this message came from her. With MACs, a third party could only be convinced of integrity and authenticity if they knew the shared key, which they should hopefully not know.

        Furthermore, MACs are faster to compute, since the computation required for public-key private-key cryptography is more expensive.

        
    \end{enumerate}

    \item

      Suppose an attacker observes a pair $(M, C_K(M))$.

      When we create the $k$ blocks in computation of $C_K(M)$, we compute it from $M||0^p$, where $p$ is calculated as described.

      Assuming $p \neq 0$, i.e. $|M|$ is not a multiple of 64, then we can use $M' = M||0$.

      The $k$ blocks created in computation of $C_K(M')$ will be $M'||0^{p'}$. Since $p' = p-1$, this is equivalent to $M||0^p$, so we get $C_K(M') = C_K(M)$, so we can simply use the pair $(M', C_K(M))$

      \vspace{15pt}

      Another attack relies on the misuse of the message as the key for $E$.

      Since an attacker knows each $M_k$ from $M$, they can decrypt each $E_{K_i}(\ldots)$, and thus retrieve $K$.

      At this point, they can run the algorithm themselves on any arbitrary $M' \neq M$ to obtain $(M', C_K(M'))$

      \item
        \begin{enumerate}[label=(\roman*)]

          \item
            We can compute $H(M||K) = H_1||H_2$, where each $H_i$ is 128-bits long.

            Then, we can output $H_1 \oplus H_2$ as $C_K(M)$.

            A verifier can perform the same computation and verify that the MAC is correct.

            An adversary cannot forge a signature by computing $H(M'||K)$ since they cannot know $K$, and since $H$ is a secure hash function (not malleable), they cannot change bits of $C_K(M)$ to produce $C_K(M')$ for an $M'$ they can know, since this would require the hash function to not be pre-image resistant.

            \item
              First, we produce $K' = K||K$, to make it 128 bits.

              Then, we pad $M$ with a 1-bit and then with zeros to be a multiple of $256$ bits and split it into 256-bit blocks $M_1||\ldots||M_k$

              Then, we compute $E_{K'}(M_k \oplus E_{K'}(\ldots \oplus E_{K'}(M_1)\ldots)) = C$.

              Then, we output $E_{K'}(C) = C_1||C_2$, where each $C_i$ is 128-bits.

              And finally, we output $C_1 \oplus C_2$ to ensure the output is 128 bits.

              Effectively, we use ECBC-MAC.


            
        \end{enumerate}



        
    \end{enumerate}
\end{document}
