\documentclass[12pt]{article}
\usepackage[margin=1.5cm]{geometry}
\usepackage{parskip}
\usepackage{amsmath}
\usepackage{amssymb}
\usepackage{amsfonts}
\usepackage{enumitem}
\usepackage{graphicx}
\usepackage{stmaryrd}
\graphicspath{ {./images/} }


\begin{document}
\begin{enumerate}[label=(\alph*)]
  \item
    Since $h_1$ and $h_2$ are secure one-way functions, we will only be able to predict $r_n$ if the corresponding $s_n$ is an $s_n$ that has been used before, and we know that it has been used before.

    In order for an $s_n$ to be used for, we require a cycle to have been created in our hash chain for generating the $s_i$s. If a cycle is generated, i.e. we have $r_x = r_{x + y}$, we know that our cycle length is $y$, and we can deduce any $r_n$ with $n > x+y$ by simply observing the hashes in between $r_x$ and $r_{x+y}$.

    So we simply need to find the probability that a cycle in our hash has occurred after $n$ hashes, with an output space of size $|G|$.

    We can see this as almost analogous to our birthday problem. If we assume that $|G|$ is reasonably large, then we will find that the probability of a collision is very small before $\sqrt{|G|}$, and then quickly jumps up to near 1 after we reach $\sqrt{|G|}$.

    So we use the following probability function:

    $p = \begin{cases}0 & n < \sqrt{|G|}\\1 & n \geq \sqrt{|G|}\end{cases}$.

  \item
    \begin{enumerate}[label=(\roman*)]
      \item
        In order for $u$ and $v$ to be reasonably described as one way functions, we need to ensure that they are generates in $Z^*_p$. Since this group is of size $p-1$, this is not trivial (as it is for groups of size $p$). This means that inverting $x \mapsto u^x$ becomes as hard as the discrete logarithm problem, which we assume to be hard for large $p$, as we have.

        To generate suitable $u$, $v$, we can choose $p$ such that $p-1 = 2q$, where $q$ is a large prime.

        This means that we have $q$ quadratic residues. If we take the quadratic residues as a subgroup, then since $q$ is prime we get that any element is a generator.

        So, by using a group that is the quadratic residues of $Z^*_{p}$ with an appropriate $p$, then any element is a generator.

      \item
        Assuming that we are using the quadratic residues, as described earlier, then it is easy to determine if the output of our pseudo-random generator is a quadratic residue if $f$ is the identity function using Euler's criterion.

        Since we have $q$ quadratic residues, then after seeing $n$ outputs of our pseudo-random generator, then if it was random we would expect to see all quadratic residues with probability $\frac{1}{2^n}$, so we win with non-negligible probability.

        By letting $f(x) = x \mod{2^{2048}}$, we ensure that not all outputs will be quadratic residues of our group

      \item
        Suppose that $v = u^{e} \mod{p}$. This means that $h_2(x)$ becomes $f(u^{ex} \mod{p})$.

        Or, that $h_2(x) = h_1(ex)$

        This means that $r_i = h_1(s_i)$, and $s_{i+1} = h_1(es_i)$

        This means, however, that since Mallory knows $e$, they can compute the internal state $s_{i+1}$ after seeing $r_i$, by simply having $r_1^e$, which is equivalent to:

        $((u^x \mod {p}) \mod{2^{2048}})^e$

        We can move the exponent inside to get:

        $(u^{ex} \mod {p}) \mod {2^{2048}}$, which is equivalent to $h_2(x)$, which is exactly what is needed to compute $s_{i+1}$, so Mallory has a backdoor.








        
    \end{enumerate}
        
\end{enumerate}
\end{document}
