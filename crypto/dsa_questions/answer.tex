\documentclass[12pt]{article}
\usepackage[margin=1.5cm]{geometry}
\usepackage{parskip}
\usepackage{amsmath}
\usepackage{amssymb}
\usepackage{amsfonts}
\usepackage{enumitem}
\usepackage{graphicx}
\usepackage{stmaryrd}
\graphicspath{ {./images/} }


\begin{document}

In these questions, we consider whether the signature algorithm on slide 221 of the lecture notes would still be (a) functional and (b) secure, under the given changes.

We provide the algorithm as below:

Let $\mathbb{G}, q ,g$ be scheme-wide choices that are a cyclic group $\mathbb{G}$ of order $q$ with generator $g$, and let $H : \{0,1\}^* \rightarrow \mathbb{Z}_q$ be a collision resistant hash function, and $F : \mathbb{G} \rightarrow \mathbb{Z}_q$

Let $x \in \mathbb{Z}_q$ be a secret key, and public key be $y = g^x \in \mathbb{G}$

Then, to sign a message $m \in \{0,1\}^*$, we choose a random $k \in \mathbb{Z}_q^*$ and set $r = F(g^k)$

Then, solve for $s$ in $ks - xr \equiv H(m) \pmod{q}$

If $r=0$ or $s=0$, restart with a different $k$.

Finally, output $(r,s)$ as the signature.

To verify, consider with a valid signature:

$g^{ks - xr} = g^{H(m)}$

$(g^k)^s = g^{H(m)}y^r$

$g^k = g^{H(m)s^{-1}}y^{rs^{-1}}$

$r = F(g^{H(m)s^{-1}}y^{rs^{-1}})$

So we simply verify whether $r$ matches the expression on the RHS.



    \begin{enumerate}
      \item

        In this scenario, suppose the signer uses the same $k$ to sign two different messages.

        \begin{enumerate}

          \item
            The scheme remains functional, since verification is stateless. Essentially, re-using the same $k$ does not stop any certificates from being verified, or make any false certificates verifiable.
          \item
            The scheme does not remain secure.

            Suppose we generate two signatures $(r_1, s_1)$ and $(r_2, s_2)$ for messages $m_1, m_2$, both using the same $k$.

            Since $r = F(g^k)$, we actually get that $r_1 = r_2$ (so we refer to both of them just by $r$ now)

            So we know that $s_1 = k^{-1} \cdot (H(m_1) + xr) \mod q$

            And $s_2 = k^{-1} \cdot (H(m_2) + xr) \mod q$

            So $s_1 - s_2 = k^{-1} \cdot (H(m_1) + H(m_2)) \mod q$

            Thus, $k = \frac{H(m_1) + H(m_2)}{s_1 - s_2} \mod q$

            Which we can solve easily.

            Furthermore, we can then solve for $x$ in either equation, completely breaking the scheme.
        \end{enumerate}

      \item
        In this scenario, suppose $H$ is not collision resistant.

        \begin{enumerate}
          \item
            The scheme remains functional, since functionality of the scheme does not depend on collision resistance, the functionality just depends on $H$ generating a member of $\mathbb{Z}_q$.

          \item
            The scheme does not remain secure, however, since someone could generate two $m_1 \neq m_2$ such that $H(m_1) = H(m_2)$. Then they could generate a signature for $m_1$ (which is really a signature for $H(m_1)$), and then later substitute $m_1$ for $m_2$, with the signature still being verifiable.
        \end{enumerate}

      \item
        In this scenario, suppose $F$ is not collision resistant.

        \begin{enumerate}
          \item
            The scheme remains functional. Commonly, we choose $F$ to not even be pre-image resistant. We can see that even if $F$ is the identity (with $\mathbb{G} = \mathbb{Z}_q$), the algorithm still works as expected.

          \item
            The scheme also remains secure. If $F$ was not collision resistant, an attacker would just be able to generate a different group element $l$ such that $F(g^k) = F(l)$, but since $k$ is never divulged, or known, we cannot attack the scheme with this.
        \end{enumerate}

      \item
        In this scenario, suppose $k$ and $r$ are omitted, so a signature is just $s$ satisfying $x \cdot s \equiv H(m) \pmod {q}$.

        \begin{enumerate}
          \item
            In this case, to verify, we check that $g^{H(m)s^{-1}} = y$

            The scheme remains functional, since we still have a verification condition that will verify the signature of a message $m$, and does not verify every other message.

          \item
            The scheme does not remain secure, however. Since $xs \equiv H(m) \pmod{q}$, we get $x \equiv H(m) \cdot s^{-1} \pmod{q}$, so we can easily solve for $x$, the private key, rendering the scheme insecure.
        \end{enumerate}

      \item

        In this scenario, suppose $r$ is generated randomly rather than as a function of $k$.

        \begin{enumerate}
          \item
            The scheme is no longer functional, since the verification that works by using everything as an exponent to $g$ no longer works, because there is simply nothing to verify, since we do not know $g^k$. (If $k$ is published so we know $g^k$, we can recover $x$).

          \item
            The scheme still remains secure (assuming secure means $x, k$ is not leaked), by appeal to the security of our original algorithm, since we make no changes that could reveal either number.
        \end{enumerate}

      \item
        In this scenario, suppose we do not check for $r=0$ or $s=0$ special cases.

        \begin{enumerate}
          \item
            The scheme is not functional, since the verifier must compute $s^{-1}$, and this is not possible if $s=0$. If only $r=0$, however, then the scheme remains functional, since all of the equations still hold.

          \item
            The scheme does not remain secure. If $s=0$, then an attacker can recover $x$ (as seen in scenario 4). If the verifier accepts $r=0$, then we can create a signature for any message by simply choosing $s = k^{-1} H(m)$ with an arbitrary $k$, without needing to know $x$.
        \end{enumerate}

      \item
        In this scenario, suppose we require that $ks + xr \equiv H(m) \pmod{q}$ instead if $ks - xr \equiv H(m) \pmod{q}$.

        \begin{enumerate}
          \item
            We see that:

            $g^{ks + xr} = g^{H(m)}$

            $g^{ks} = g^{H(m)}y^{-r}$

            $g^k = g^{H(m)s^{-1}}y^{-rs^{-1}}$

            So the verifier is still functional, since we can compute the term on the RHS to verify against $r$.

          \item
            The verifier remains secure, also, since this is equivalent to picking $F' = -F$, so the security properties still all hold.
        \end{enumerate}

      \item

        In this scenario, suppose we require that $ks \cdot xr \equiv H(m) \pmod{q}$ instead of $ks - xr \equiv H(m) \pmod{q}$.

        \begin{enumerate}
          \item
            Now, we see that:

            $g^{ksxr} = g^{H(m)}$

            $g^{k} = g^{H(m)s^{-1}x^{-1}r^{-1}}$

            Since the verifier does not know $x$, they cannot compute this.

            Another strategy might be to view it as:

            $y^{ksr} = g^{H(m)}$

            $y = g^{H(m)k^{-1}s^{-1}r^{-1}}$

            But since the verifier does not know $k$, they cannot compute this either.

          \item
            We now get that $s := k^{-1}x^{-1}r^{-1}H(m) \mod q$

            This means that after an attacker learns a signature, they know $s, r$ and thus can compute $kx = H(m)s^{-1}r^{-1}$

            Since they know $kx$, they can forge signatures for any message without knowing $x$ by reusing $r' = r$ and setting $s' = H(m')(kx)^{-1}r^{-1}$
        \end{enumerate}

      \item
        In this scenario, suppose we use $\mathbb{Z}_q$ under addition as $\mathbb{G}$.

        \begin{enumerate}
          \item
            The scheme remains functional, since $\mathbb{G}$ is still a valid group, and the algorithm simply requires that, the details of $\mathbb{G}$ are abstracted away.

          \item
            The scheme does not remain secure, however.

            $g^x$ becomes equivalent to $xg \mod q$, so since we know $y = xg \pmod{q}$, we can compute $x = yg^{-1} \pmod{q}$ easily.

            Essentially, the scheme is not secure because the discrete logarithm problem becomes easy.
        \end{enumerate}

      \item

        In this scenario, suppose $q$ is not prime.

        \begin{enumerate}
          \item
            The scheme remains functional. $q$ not being prime simply means that it may not be trivial to find a generator $g$, but a generator may still exist. We have to be careful of when $k,q$ are not coprime, since then there does not exist a $k^{-1}$, and so we must generate a different $k$, but apart from that we retain functionality.

          \item
            The scheme is not trivially breakable, but is less secure, since Pohlig-Hellman can be used to solve the discrete log problem in time proportional the largest prime factor of $q$.
        \end{enumerate}

      \item
        In this scenario, suppose $g$ is not a generator of $\mathbb{G}$.

        \begin{enumerate}
          \item
            The scheme remains functional. Since $g = g'^l$ for some real generator $g'$ and natural number $l$, we get that we are effectively working in the $l$th residues of $\mathbb{G}$.

            This means that our scheme is still functional, since we are still working in a subgroup.

          \item
            Again, the scheme is not trivially breakable, but is less secure, since we are effectively working in a smaller group, and security grows not with $q$ but with $\frac{q}{l}$.
        \end{enumerate}


        
    \end{enumerate}
\end{document}
