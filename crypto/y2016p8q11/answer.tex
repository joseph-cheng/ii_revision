\documentclass[12pt]{article}
\usepackage[margin=1.5cm]{geometry}
\usepackage{parskip}
\usepackage{amsmath}
\usepackage{amssymb}
\usepackage{amsfonts}
\usepackage{enumitem}
\usepackage{graphicx}
\usepackage{stmaryrd}
\graphicspath{ {./images/} }


\begin{document}
\begin{enumerate}[label=(\alph*)]

      \item

        Collision-resistant hash functions require a key $s$ and security parameter $n$ because otherwise, a theoretical adversary exists that always outputs two $x, x'$ that collide. This adversary exists because the entire operation of the hash function is known beforehand, so the adversary can be pre-constructed with this collision in mind.

        By including a key $s$ and security parameter $n$, we require that the adversary cannot rely on pre-computation to derive a collision.

      \item
        \begin{enumerate}[label=(\roman*)]
          \item
            In this case, $H_s$ is not a collision-resistant hash function, because it is not even a hash function! Hash functions require a fixed-length output, but this is a variable-length output.

          \item
            In this case, $H_s$ is a collision-resistant hash function. If it was not, then $h_s(x)$ would not be a collision-resistant hash function, which would be a contradiction:

            Suppose an adversary $A$ exists that given key $s$ can derive a collision between $x$ and $x'$ for $H_s(x)$ with negligible probability in PPT, such that $h_s(x) || LSB(x) = h_s(x') || LSB(x')$.

            This requires that $h_s(x) = h_s(x')$, so we can use adversary $A$ to find a collision between $x$ and $x'$ with negligible probability in PPT.

          \item
            In this case, $H_s$ is not a collision resistant hash function. We construct the following adversary:

            \begin{itemize}
                
              \item
            Input key $s$

              \item
            Choose $x = 0$, $x' = 1$

              \item
            Return $x, x'$
            \end{itemize}

            Since $x | 1 = 1 = x' | 1$, $H_s(x) = H_s(x')$, so we win with probability 1.
        \end{enumerate}

      \item
        Euler's theorem tells us that if $gcd(a,n) = 1$, then $a^{\varphi(n)} \equiv 1 \pmod{n}$. This means that $a^{\varphi(n) - 1}$ is the modular inverse of $a$ modulo $n$.

        In this case, $a=5$ and $n=8$, so $gcd(a,n) = 1$

        $\varphi(n) = 4$, which can be found by counting (1, 3, 5, 7)

        So we need to calculate $5^3 \pmod{8}$

        $5^3 = 125$, and $125 \pmod{8} \equiv 5$

        Therefore, $5^{-1} \mod 8 = 5$

      \item
        \begin{enumerate}[label=(\roman*)]
          \item

            The downsides of using $Z^*_p$ come from the fact that it has order $p-1$, which is not prime.

            This gives two disadvantages:

            \begin{itemize}
                \item
                  Finding a generator is hard. In a prime order group (like a prime order subgroup of $Z^*_p$), every element is a generator, so it is easy to find one.

                \item
                  We are vulnerable to the Pohlig-Hellman algorithm for computing discrete logarithms, which allows us to reduce the problem of calculating a discrete logarithm for a group of size $p-1$ to calculating discrete-logarithms for prime-order subgroups, which at least halves the time it takes to find one.
            \end{itemize}

          \item
            There exists an algorithm for computing discrete logarithms in $Z_p^*$ (and thus its subgroups) called the Index Calculus Algorithm. It has runtime $2^{O(\sqrt{\log p \log \log p})}$, which is sub-exponential. Generic algorithms are still exponential in $q$

            This runtime does not include $q$, so $p$ has to be very big to be resistant to these attacks, but $q$ does not have to be much bigger. We are still able to choose a small $q$ and be resistant from generic algorithms for DLP since they have exponential runtime. Choosing a smaller $q$ allows the bit-length of group elements to be smaller, and thus computations on the group elements are faster.

            
        \end{enumerate}



        
    \end{enumerate}
\end{document}
