\documentclass[12pt]{article}
\usepackage[margin=1.5cm]{geometry}
\usepackage{parskip}
\usepackage{amsmath}
\usepackage{amssymb}
\usepackage{amsfonts}
\usepackage{enumitem}
\usepackage{graphicx}
\usepackage{stmaryrd}
\graphicspath{ {./images/} }


\begin{document}
\begin{enumerate}[label=(\alph*)]

  \item
    \begin{enumerate}[label=(\roman*)]

      \item

        This reduces the security of a system because there a maximum of $2^{30}$ possible 128-bit random numbers that can be generated, as opposed to $2^{128}$ actually possible 128-bit random numbers. Brute forcing $2^{30}$ possible values is feasible compared to brute forcing $2^{128}$ possible values, and thus MACs could be forged.

        \item

          If we get one random number every 10 milliseconds, in half an hour we get 180000 random numbers.

          If we are generating truly random 128-bit numbers, then we only expect to see a collision within around $\sqrt{2^{128}} = 2^{64}$ numbers, by the birthday problem, whereas if we are using a tampered device, we expect to see a collision within around $\sqrt{2^{30}} = 2^{15}$ numbers, again by the birthday problem.

          $2^{64} \gg 180000$, but $2^{15} < 180000$, so we would only expect to see a collision if we are using a tampered device.

          The probability that we see a collision with a non-tampered device will be almost zero.

          The probability that we do not see a collision with a tampered device will be almost zero also.

          Therefore, after generating 180000 random numbers, we simply check if there is a collision, and if there is we guess that it is tampered, and if not then we guess that it is not tampered, which has a more than 0.5 success chance of winning.
    \end{enumerate}

    \item
      \begin{enumerate}[label=(\roman*)]

        \item
          In CFB, our encryption equation is:

          $C_{i+1} = M_{i+1} \oplus E_K(C_{i})$

          So we encrypt the previous ciphertext block with our PRP, and then XOR it with the message block.

          By rearranging, to decrypt we see that:

          $M_{i+1} = C_{i+1} \oplus E_K(C_i)$

          So we encrypt the previous ciphertext block with our PRP, and XOR it with the next ciphertext block.

          \item
            Not relevant?

            \item
              A secure hash function can be used to implement a one-time signature scheme using Lamport signatures.

              With an $n$-bit hash function $H$, we generate $2n$ random bit strings, each $n$ bits long, call them $R_{b,i}$ where $0 \leq b \leq 1$ and $0 \leq i \leq 255$.

              We then distribute $H(R_{b,i})$ for all $b,i$ as our public key.

              Suppose we want to sign message $m$.

              We first compute $x = H(m)$ to give us a 256-bit bitstring.

              Then, we produce signature $y = (R_{x_0, 0}, \ldots, R_{x_{255}, 255})$. Essentially, we use each bit of $x$ to choose a bitstring.

              Then, a recipient can verify the signature by computing $H(m)$, and verifying that each element $y_i$ when hashed gives $H(y_i) = H(R_{x_i, i})$ from the public key.

              \item
                If the same private key is used multiple times to sign different messages, then the scheme loses security as more and more of the private key $R$ is leaked, since the signature leaks parts of the private key. After just one signature, an adversary could only forge a signature if they knew a message $m$ with the same hash. However, as more bits get leaked, there are more possibilities of $m$, as the attacker knows more of the bitstrings in $R$.
          
      \end{enumerate}
        
    \end{enumerate}
\end{document}
