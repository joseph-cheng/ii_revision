\documentclass[12pt]{article}
\usepackage[margin=1.5cm]{geometry}
\usepackage{parskip}
\usepackage{amsmath}
\usepackage{amssymb}
\usepackage{amsfonts}
\usepackage{enumitem}
\usepackage{graphicx}
\usepackage{stmaryrd}
\graphicspath{ {./images/} }


\begin{document}
\begin{enumerate}[label=(\alph*)]
  \item
    \begin{enumerate}[label=(\roman*)]
      \item
        
    In GF($2^8$), subtraction is simply XOR, so we calculate the difference as follows:

    \texttt{0101 1001}

  \item
    To calculate such a product, we perform normal binary multiplication, and then reduce modulo \texttt{1 0001 0011}

\begin{verbatim}
     0100 1011 x
     0000 1001
--------------
     0100 1011 +
 010 0101 1000
 010 0001 0011
\end{verbatim}

We then perform polynomial division to reduce \texttt{010 1010 0011} modulo \texttt{1 0001 0011}:

\begin{verbatim}

--------------
|010 0001 0011
 -10 0010 0110
  00 0011 0101
            
\end{verbatim}

This fits in our 8-bits, so our final value is \texttt{0011 0101}

    \end{enumerate}
\item
  There are two main advantages of $GF(2^{128})$ over $\mathbb{Z}_{2^{128}}$.

  \begin{itemize}
    \item
      Firstly, implementing arithmetic in $GF(2^{128})$ is much simpler. We do not require carry bits for addition/subtraction, so we can parallelise the whole process, as opposed to in $\mathbb{Z}_{2^{128}}$ where we require carries to propagate through, making parallelising hard, and implementation logic harder.

    \item
      Secondly, since $2^{128}$ is not prime, $(\mathbb{Z}_{2^{128}}, +, \cdot)$ is not a field, because we do not get multiplicative inverses for every element in $\mathbb{Z}^*_{2^{128}}$, which means that we cannot do division, whereas $GF(2^{128})$ is a field so we can do division. Division ends up being useful for certain applications (not sure where though?).

  \end{itemize}

\item
  For CBC, our encryption equation is $C_i = E_K(C_{i-1} \oplus M_i)$, so the corresponding decryption equation is $M_i = D_K(C_i) \oplus C_{i-1}$

  In CBC, we can calculate each $D_K$ in parallel, since we know each of the $C_i$ values at the start.

  For OFB, our encryption equation is $C_i = M_i \oplus R_i$, where $R_i$ are generated as $R_0 = C_0$ and $R_{i+1} = E_K(R_{i})$, so our decryption equation is $M_i = C_i \oplus R_i$

  In OFB, we cannot calculate the $E_K$ in parallel, since we require the previous $R$ to calculate the next $R$.

  For CTR, our encryption equation is $C_i = M_i \oplus E_K(T_i)$, where $T_i = O+i$ for some private offset $O$, so our decryption equation is $M_i = C_i \oplus E_K(T_i)$

  In CTR, we can parallelise the $E_K$ calls, since we learn $O$ from $C_0$, and then we know each $i$ that we need also.





        
\end{enumerate}
\end{document}
