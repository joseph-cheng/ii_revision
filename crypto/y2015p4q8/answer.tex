\documentclass[12pt]{article}
\usepackage[margin=1.5cm]{geometry}
\usepackage{parskip}
\usepackage{amsmath}
\usepackage{amssymb}
\usepackage{amsfonts}
\usepackage{enumitem}
\usepackage{graphicx}
\usepackage{stmaryrd}
\graphicspath{ {./images/} }


\begin{document}
\begin{enumerate}[label=(\alph*)]

    \item
        A pseudo-random generator is a function $\{0,1\}^n \rightarrow \{0,1\}^{e(n)}$ (where $e(n) > n$), such that no distinguisher exists that can distinguish between a single output of a PRG, and a single output of a truly random number.

        A (keyed) pseudo-random function is a function $F : \{0,1\}^k \rightarrow \{0,1\}^m \rightarrow \{0,1\}^n$, such that no distinguisher exists that can distinguish between outputs from $F_K$ (given oracle access), and a function $f$ chosen at random from the set of all functions $\{0,1\}^m \rightarrow \{0,1\}^n$.

        The key difference here between the security definitions is that a PRG is only required to generate a single indistinguishable number, whereas a PRF is required for the entire function to be indistinguishable from a random function.

    \item
        Not relevant
    \item
        Existential unforgeability is a security definition for MACs that essentially requires that no adversary can construct a MAC for a message on its own, given oracle access to the MAC function.

        We describe the game as follows:

        The user $\mathcal{U}$ chooses a key $K$

        An adversary $\mathcal{A}$ is able to send a message $M_i$ to $\mathcal{U}$. $\mathcal{U}$ must reply with tag $T_i = MAC_K(M_i)$

        The previous step can be completed a polynomial number of times.

        $\mathcal{A}$ then sends a pair $(M,T)$ to $\mathcal{U}$, with $M \neq M_i$ for all $i$

        The adversary wins if $Vrfy_K(M,T) = 1$, and loses otherwise.

        We say a MAC has existential unforgeability if no adversary exists that wins this game with non-negligible probability.

    \item
        The problem with CBC-MAC is that it only provides existential unforgeability for fixed-length messages. If variable-length messages are allowed, an adversary can send $M_1$ as arbitrary, and receive $T_1 = E_K(M_1)$. Then, they can maliciously choose $M = M_1||(T_1 \oplus M_1)$. Since $CBC-MAC_K(M)= E_K((T_1 \oplus M_1) \oplus E_K(M_1)) = E_K((T_1 \oplus M_1) \oplus T_1) = E_K(M_1) = T_1$, we can choose $T = T_1$ and win with probability 1.

        ECBC-MAC solves this with adding an additional use of $E$: $ECBC-MAC_{K,M'}(M) = E_{K'}(CBC-MAC_K(M))$

        This stops the length-extension attack, since the output from ECBC-MAC is no longer what gets fed into the next block.

    \item
        Not relevant



        
    \end{enumerate}
\end{document}
