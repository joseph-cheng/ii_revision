\documentclass[12pt]{article}
\usepackage[margin=1.5cm]{geometry}
\usepackage{parskip}
\usepackage{amsmath}
\usepackage{amssymb}
\usepackage{amsfonts}
\usepackage{enumitem}
\usepackage{graphicx}
\usepackage{stmaryrd}
\graphicspath{ {./images/} }


\begin{document}
\begin{enumerate}[label=(\alph*)]
  \item
    Beta-reduction is a relation between two expressions in the PLC.

    We have that $e \rightarrow_\beta e'$ if $e$ is of the form $(\lambda x. e_1) e_2$, and $e'$ is of the form $e_1[e_2 / x]$, or $e$ is of the form $(\Lambda \alpha (e_1)) T$ and $e'$ is of the form $e_1[T / \alpha]$, or that a sub-expression of $e$, $e_1$ beta-reduces to $e_1'$, and $e'$ is the result of beta-reducing $e_1$ in $e$.

    Beta-conversion is the equivalence relation generated by beta-reduction.

    A beta-normal form is a term in the PLC that does not beta-reduce to any other term.

    Typeable PLC expressions are beta-convertible to beta-normal forms that are unique up to alpha-conversion because of two reasons:

    \begin{itemize}
      \item
        The Church-Rosser theorem tells us that if $e \rightarrow_\beta^* e_1$ and $e \rightarrow_\beta^* e_2$, then there exists some $e'$ such that $e_1 \rightarrow_\beta^* e'$ and $e_2 \rightarrow_\beta^* e'$

      \item
        The strong normalization property of the PLC tells us that typeable PLC expressions never have result in an infinite chain of beta-reductions.
    \end{itemize}

    The second fact tells us that all expressions $e$ reduce to some beta-normal form $e'$, and the first fact tells us that if $e$ reduces to two beta-normal forms, $e'$ and $e''$, then by the Church-Rosser theorem they must be equivalent up to $\alpha$-equivalence.

    The same is not true for untypeable PLC expressions, since the strong normalization property does not hold. For example, we can construct the term $\Omega = \Lambda \alpha ((\lambda x : \alpha. x x) (\lambda x : \alpha. x x))$, which always reduces to itself, and thus has no beta-normal form.

  \item
    \begin{enumerate}[label=(\roman*)]
      \item
        We give:

        $\Lambda \alpha (\lambda x : \alpha (\Lambda \beta (\lambda f : \alpha \rightarrow \beta (f x))))$

        We give the following typing derivation, letting $\Gamma = \{f : \alpha \rightarrow \beta, x : \alpha\}, \Theta = \{\alpha, \beta\}$.

        $\Theta;\Gamma \vdash f : \alpha \rightarrow \beta$ (by var)

        $\Theta;\Gamma \vdash x : \alpha$ (by var)

        $\Theta;\Gamma \vdash f x : \beta$ (by app)

        $\Theta;\{x : \alpha\} \vdash \lambda f : \alpha \rightarrow \beta (f x) : (\alpha \rightarrow \beta) \rightarrow \beta$ (by fn)

        $\{\alpha\};\{x : \alpha\} \vdash \Lambda \beta (\lambda f : \alpha \rightarrow \beta (f x)) : \tau$ (by $\forall$ fn)

        $\{\alpha\};\{\} \vdash \lambda x : \alpha (\Lambda \beta (\lambda f : \alpha \rightarrow \beta (f x))) : \alpha \rightarrow \tau$ (by fn)

        $\{\};\{\} \vdash I : \forall \alpha (\alpha \rightarrow \tau)$ (by $\forall$ fn)

        As required.

      \item
        We give:

        $\Lambda \alpha (\lambda t : \forall \beta ((\alpha \rightarrow \beta) \rightarrow \beta) (t \alpha (\lambda x : \alpha (x))))$

        We give the following typing derivation, letting $\Gamma = \{t : \tau\},  \Theta = \{\alpha\}$

        $\Theta;\Gamma \vdash t : \forall \beta ((\alpha \rightarrow \beta) \rightarrow \beta)$ (By var)

        $\Theta \vdash \alpha type$

        $\Theta;\Gamma \vdash t \alpha : (\alpha \rightarrow \alpha) \rightarrow \alpha$ (by $\forall$ app)

        $\Theta;\Gamma,x:\alpha \vdash x : \alpha$ (by var)

        $\Theta;\Gamma \vdash \lambda x : \alpha(x) : \alpha \rightarrow \alpha$ (by fn)

        $\Theta;\Gamma \vdash t \alpha (\lambda x : \alpha (x)) : \alpha$ (by app)

        $\Theta;\{\} \vdash \lambda t : \tau (t \alpha (\lambda x : \alpha (x))) : \tau \rightarrow \alpha$ (by fn)

        $\{\};\{\} \vdash J : \forall \alpha (\tau \rightarrow \alpha)$ (by $\forall$ fn)

        As required.

      \item
        We give the following steps of beta-reduction:

        $I \alpha x \rightarrow_\beta (\lambda x : \alpha (\Lambda \beta (\lambda f : \alpha \rightarrow \beta (f x)))) x$

        $\rightarrow_\beta \Lambda \beta (\lambda f : \alpha \rightarrow \beta (f x))$

        $J \alpha (I \alpha x) \rightarrow_\beta (\lambda t : \forall \beta((\alpha \rightarrow \beta) \rightarrow \beta(t \alpha (\lambda x : \alpha (x))))) (I \alpha x)$

        $\rightarrow_\beta^3 (\Lambda \beta (\lambda f : \alpha \rightarrow \beta (f x))) \alpha (\lambda x : \alpha (x))$

        $\rightarrow_\beta (\lambda f : \alpha \rightarrow \alpha (f x)) (\lambda x : \alpha(x))$

        $\rightarrow_\beta (\lambda x : \alpha (x)) x$

        $\rightarrow_\beta x$

        Therefore, we clearly see that:

        $\Lambda \alpha (\lambda x : \alpha (J \alpha (I \alpha x)))$ has beta-normal form $\Lambda \alpha (\lambda x : \alpha (x))$.



        
    \end{enumerate}

  \item
    This has beta-normal form $\Lambda \alpha (\lambda y: \tau (y))$

        
\end{enumerate}
\end{document}
