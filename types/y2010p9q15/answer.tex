\documentclass[12pt]{article}
\usepackage[margin=1.5cm]{geometry}
\usepackage{parskip}
\usepackage{amsmath}
\usepackage{amssymb}
\usepackage{amsfonts}
\usepackage{enumitem}
\usepackage{graphicx}
\usepackage{stmaryrd}
\graphicspath{ {./images/} }


\begin{document}
\begin{enumerate}[label=(\alph*)]
    \item

        $Z$ is of type $\forall \alpha (\alpha \rightarrow (\alpha \rightarrow \alpha) \rightarrow \alpha)$, or $nat$

        $S$ is of type $nat \rightarrow nat$

    \item
        We define $I$ as follows:

        $\Lambda \alpha(\lambda x : \alpha (\lambda f: \alpha \rightarrow \alpha (\lambda n : nat (n \alpha x f))))$

        We see that $I \alpha x f Z$ reduces to $Z \alpha x f$, which by definition of $Z$ reduces to $x$, as required.

        We see that $I \alpha x f (S y)$ reduces to $(S y) \alpha x f$, which by definition of $S$ reduces to $f (y \alpha x f)$, and from before we know that $y \alpha x f$ is beta-convertible to $I \alpha x f y$, so this is all convertible to $f (I \alpha x f y)$, as required.

    \item
        The beta-normal form of $S^0 Z$ is just $Z$, since $S^0 Z = Z$, and $Z$ is beta-normal.

        $S^1 Z = S Z$, so this beta-reduces to $\Lambda \alpha(\lambda x : \alpha (\lambda f : \alpha \rightarrow \alpha (f (Z \alpha x f))))$, which reduces to  $\Lambda \alpha (\lambda x : \alpha (\lambda f : \alpha \rightarrow \alpha (f x)))$

        By similar reasoning, $S^2 Z$ reduces to $\Lambda \alpha(\lambda x : \alpha (\lambda f : \alpha \rightarrow \alpha (f ((S Z) \alpha x f))))$, which reduces to $\Lambda \alpha(\lambda x : \alpha (\lambda f : \alpha \rightarrow \alpha (f (f x))))$

        In general, we can see that the beta-normal form of $S^n Z$ is $\Lambda \alpha (\lambda x : \alpha (\lambda f : \alpha \rightarrow \alpha (f^n x)))$, where $f^n$ is defined analogously to $S^n$.

    \item
        \begin{enumerate}[label=(\roman*)]
            \item
                We define $A$ as follows:

                $\lambda n : nat (\lambda m : nat (I nat n S m))$.

                This essentially works by iteratively applying $S$ to $n$, $m$ number of times, giving us $S^{n+m}Z$

                From the types of $I$ and $S$, we easily see this has type $nat \rightarrow nat \rightarrow nat$ as required.

                To prove the required property of $A$, we use induction over the natural numbers.

                Base case $m=0$:

                $A(Z)(S^n Z)$ reduces to $I nat (S^n Z) S Z)))$, and by definition of $I$ we get this reduces to $S^n Z$, as required.

                Inductive case, assume the result holds for $m=k$, try $m=k+1$:

                $A(S^{k+1}Z)(S^n Z)$ reduces to $I nat (S^nZ) S (S^{k+1}Z)$, which by definition of $I$ and $S^{k+1}$ converts to $S(I \alpha (S^nZ) S (S^k Z))$. By induction, we have that $I \alpha (S^n Z) S (S^k Z)$ converts to $S^{n+k}Z$, so the whole thing converts to $S (S^{n+k} Z)$, which is $S^{n+k+1} Z$.

                Therefore, by induction, we are done.

            \item
                Again, through the types of $A$ and $I$, $M$ has the required type.

                We again proceed by induction over the natural numbers to prove the required property.

                Base case $n=0$:

                $M(S^m Z)(Z)$ reduces to $I nat Z (A (S^m Z)) Z$

                By definition of $I$ this reduces to $Z$, which is the same as $S^{m \cdot 0} Z$.

                Inductive case: assume the result holds for $n=k$, try $n=k+1$:

                $M(S^m Z)(S^{k+1}Z)$ reduces to $I nat Z (A (S^m Z)) (S^{k+1} Z)$

                Again, by definition of $I$ and $S^{k+1}$ this converts to:

                $(A (S^m Z)) (I nat Z (A (S^m Z)) S^k Z)$

                By induction, we have that $I nat Z (A (S^m Z)) S^k Z$ converts to $S^{k \cdot m} Z$, so the whole thing converts to:

                $(A (S^m Z)) (S^{k\cdot m} Z)$, which from (i) converts to $S^{(k+1) \cdot m} Z$, as required.

                Therefore, by induction, $M$ has the required property.




            
        \end{enumerate}



        
    \end{enumerate}
\end{document}
