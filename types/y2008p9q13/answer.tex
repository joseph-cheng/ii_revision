\documentclass[12pt]{article}
\usepackage[margin=1.5cm]{geometry}
\usepackage{parskip}
\usepackage{amsmath}
\usepackage{amssymb}
\usepackage{amsfonts}
\usepackage{enumitem}
\usepackage{graphicx}
\usepackage{stmaryrd}
\graphicspath{ {./images/} }


\begin{document}
\begin{enumerate}[label=(\alph*)]
  \item
    The Curry-Howard correspondence is a correspondence between elements of a programming language and mathematical logic. In this particular case, we look at the PLC, and 2IPC.

    With this correspondence, we have that types correspond to formulas. Ideas like universal quantification in 2IPC correspond to quantification over types in PLC. Implication corresponds to functions.

    Programs correspond to proofs. In particular, a program with type $T$ corresponds to a proof of the corresponding formula for $T$. For example with implication, a function of type $T \rightarrow T'$ tells us that if we have a program of type $T$, we can obtain a program of type $T'$. Analogously, if we know $A \Rightarrow B$, and we know that $A$ is true, we can deduce that $B$ must be true.

    Conjunction and disjunction corresponds with product and sum types respectively, which we can encode with the PLC, e.g. the encoding of $p \vee q$ is given in the question.

    Furthermore, $\beta$ reduction in the PLC corresponds to simplification of a proof in 2IPC.

    We now give the typing rules for the PLC, which correspond to deduction rules in 2IPC:

    \begin{align*}
      \textsc{var}&\frac{}{\Theta;\Gamma \vdash x : T}(\Gamma(x) = T)\\
      \textsc{fn}&\frac{\Theta;\Gamma, x : T \vdash e : T'}{\Theta;\Gamma \vdash \lambda x : T.e : T\rightarrow T'}\\
      \textsc{app}&\frac{\Theta;\Gamma \vdash e : T \rightarrow T' \hspace{15pt} \Theta;\Gamma \vdash e' : T}{\Theta;\Gamma \vdash e e' : T'}\\
      \textsc{$\forall$ fn}&\frac{\Theta, \alpha;\Gamma \vdash e : T}{\Theta;\Gamma \vdash \Lambda \alpha(e) : \forall \alpha (T)}\\
      \textsc{$\forall$ app}&\frac{\Theta;\Gamma \vdash e : \forall \alpha (T) \hspace{15pt} \Theta \vdash T' type}{\Theta;\Gamma \vdash e T' : T[T' / \alpha]}\\
    \end{align*}

    To give a proof of $\{\} \vdash \forall p,q,r ((p \rightarrow r) \rightarrow (q \rightarrow r) \rightarrow (p \vee q) \rightarrow r$, we simply need to find a program that has a corresponding type.

    We use the program given below.

    Let $\Gamma = \{x : p \rightarrow r, y : q \rightarrow r, z : p \vee q\}, \Theta = \{p,q,r\}$

    We get the following typing derivation:

    $\Theta;\Gamma \vdash z : \forall r((p \rightarrow r) \rightarrow (q \rightarrow r) \rightarrow r)$ (by var)

    $\Theta \vdash r type$

    $\Theta;\Gamma \vdash z r : (p \rightarrow r) \rightarrow (q \rightarrow r) \rightarrow r$ (by $\forall$ app)

    $\Theta;\Gamma \vdash x : p \rightarrrow r$ (by var)

    $\Theta;\Gamma \vdash z r x : (q \rightarrow r) \rightarrow r$ (by app)

    $\Theta;\Gamma \vdash y : q \rightarrow r$ (by var)

    $\Theta;\Gamma \vdash z r x y :r$ (by app)

    $\Theta;\{\} \vdash \lambda x : p \rightarrow r, y : q \rightarrow r, z : p \vee q (z r x y) : (p \rightarrow r) \rightarrow (q \rightarrow r) \rightarrow (p \vee q) \rightarrow r$ (through 3 applications of fn)

    $\{\};\{\} \vdash \Lambda p,q,r (\lambda x : p \rightarrow r, y : q \rightarrow r, z : p \vee q (z r x y)) : \forall p,q,r((p \rightarrow r) \rightarrow (q \rightarrow r) \rightarrow (p \vee q) \rightarrow r)$

    As required.

  \item
    As mentioned before, $\beta$ reduction on PLC expressions can be used to simplify proofs in 2IPC.

    Suppose we have a program $e$ that is a proof for some type $T$.

    Since $e$ is a proof for a type $T$, it is typeable. We then use the following properties of the PLC:

    \begin{itemize}
      \item
        Type safety: this tells us that if a program $e$ has type $T$, and $e \rightarrow_\beta e'$, then $e'$ has type $T$.

      \item
        Strong normalization: this tells us that no typeable programs result in an infinite chain of $\beta$-reduction.

      \item
        The Church-Rosser theorem: this tells us that if $e \rightarrow_\beta^* e_1$ and $e \rightarrow_beta^* e_2$, then there exists some $e'$ such that $e_1 \rightarrow_\beta^* e'$ and $e_2\rightarrow_\beta^* e'$.
    \end{itemize}

    The last two properties give us that our program $e$ must reduce to some $\beta$ normal form $e'$, for which no beta-reduction is possible (which we assume to be simpler than $e$).

    Combined with the first property, we get that this $e'$ must have type $T$, and therefore by $\beta$ reducing a term until it reaches its $\beta$ normal form, we are able to obtain a simpler expression that is still a proof for our original formula.
        
\end{enumerate}
\end{document}
