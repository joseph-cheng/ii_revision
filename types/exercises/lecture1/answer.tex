\documentclass[12pt]{article}
\usepackage[margin=1.5cm]{geometry}
\usepackage{parskip}
\usepackage{amsmath}
\usepackage{amssymb}
\usepackage{amsfonts}
\usepackage{enumitem}
\usepackage{graphicx}
\usepackage{stmaryrd}
\graphicspath{ {./images/} }


\begin{document}
\begin{enumerate}
  \item

    We give the following operational semantics rules:

    \begin{align*}
      \textsc{LeqCongL}&\frac{e_1 \leadsto e_1'}{e_1 \leq e_2 \leadsto e_1' \leq e_2}\\
      \textsc{LeqCongR}&\frac{e_2 \leadsto e_2'}{n \leq e_2 \leadsto n \leq e_2'}\\
      \textsc{Leq}&\frac{}{n_1 \leq n_2 \leadsto b}(b = \text{true if }n_1 \leq n_2, \text{ false otherwise})\\
      \textsc{PlusCongL}&\frac{e_1 \leadsto e_1'}{e_1 + e_2 \leadsto e_1' + e_2}\\
      \textsc{PlusCongR}&\frac{e_2 \leadsto e_2'}{n + e_2 \leadsto n + e_2'}\\
      \textsc{Plus}&\frac{}{n_1 + n_2 \leadsto n}(n = n_1 + n_2)\\
    \end{align*}

  \item
    Below, we extend the proof of progress to cover $e \wedge e'$

    By assumption, $e \wedge e'$ is typeable under the empty context, so we have:

    \[
     \frac{\{\} \vdash e : \text{bool} \hspace{15pt} \{\} \vash e' : \text{bool}}{\{\} \vdash e \wedge e' : \text{bool}}\\
    \]

    By structural induction, we therefore have that progress applies to $e$ and $e'$.

    Therefore, we know that $e$ is either a value, or there exists $e_1$ such that $e \leadsto e_1$

    We proceed by cases:

    \begin{itemize}
      \item
        If $e$ is a value, then since it is a bool, it is either true or false.

        Cases again:

        \begin{itemize}
          \item
            $e = true$

            If this is the case, we use the \textsc{AndTrue} rule to derive $e \wedge e' \leadsto e'$

          \item
            $e = false$

            If this is the case, we use the \textsc{AndFalse} rule to derive $e \wedge e' \leadsto false$

        \end{itemize}

      \item
        If there exists an $e_1$ such that $e \leadsto e_1$, we use the \textsc{AndCong} rule to derive $e \wedge e' \leadsto e_1 \wedge e'$.
    \end{itemize}

    Therefore, by structural induction, we have progress for the $e \wedge e'$ case.

  \item
    Below, we extend the proof of preservation to cover $e \wedge e'$.

    By assumption, $e \wedge e'$ is typeable under the empty context, and $e \wedge e' \leadsto e''$.

    We proceed by structural induction over $\leadsto$.

    We actually cover 3 cases: \textsc{AndCong}, \textsc{AndTrue}, and \textsc{AndFalse}:

    \begin{itemize}
      \item
        Case: \textsc{AndCong}

        If this is the case, we have the following:

        \[
          \frac{e \leadsto e_1}{e \wedge e' \leadsto e_1 \wedge e'}
        .\] 

        And:

        \[
          \frac{\{\} \vdash e : \text{bool} \hspace{15pt} \{\} \vdash e' : \text{bool}}{\{\} \vdash e \wedge e' : \text{bool}} 
        .\] 

        We get 3 subderivations:

        $e \leadsto e_1$

        $\{\} \vdash e : \text{bool}$

        $\{\} \vdash e' : \text{bool}$

        By induction on the subderivation $e \wedge e_1$ and $\{\} \vdash e : \text{bool}$, we know that $\{\} \vdash e_1 : \text{bool}$.

        Therefore, we can apply the \textsc{And} typing rule to get:

        \[
          \frac{\{\} \vdash e_1 : \text{bool} \hspace{15pt} \{\} \vdash e' : \text{bool}}{\{\} \vdash e_1 \wedge e' : \text{bool}}
        .\] 

        So, we have type preservation in this case.

      \item
        Case: \textsc{AndTrue}

        In this case, we have the following:

        \[
          \frac{}{true \wedge e' \leadsto e'}
        .\] 

        And:

      \[
        \frac{\{\} \vdash \text{true} : \text{bool} \hspace{15pt} \{\} \vdash e' : \text{bool}}{\{\} \vdash \text{true} \wedge e' : \text{bool}}$
      .\]

      We get subderivation $\{\} \vdash e' : bool$

      Therefore, we have type preservation in this case.

    \item
      Case: \textsc{AndFalse}

      In this case, we have the following:
      
        \[
          \frac{}{\text{false}  \wedge e' \leadsto \text{false} }
        .\] 

        And:

      \[
        \frac{\{\} \vdash \text{false} : \text{bool} \hspace{15pt} \{\} \vdash e' : \text{bool}}{\{\} \vdash \text{false} \wedge e' : \text{bool}}$
      .\]

      Therefore, since $\{\} \vdash \text{false} : \text{bool}$ by definition, we get type preservation.
    \end{itemize}

    Therefore, since type preservation holds in all cases, we are done.






\end{enumerate}

    
\end{document}
