\documentclass[12pt]{article}
\usepackage[margin=1.5cm]{geometry}
\usepackage{parskip}
\usepackage{amsmath}
\usepackage{amssymb}
\usepackage{amsfonts}
\usepackage{enumitem}
\usepackage{graphicx}
\usepackage{stmaryrd}
\graphicspath{ {./images/} }


\begin{document}
    \begin{enumerate}
      \item

        We give the embedding of classic into intuitionistic logic for the Godel-Gentzen translation.
        

        We only give the sum type embedding, since the embedding for the other translations is the same as the translation given in the lectures:

        $(L e)^\circ = \lambda k: (\sim A^\circ \times \sim B^\circ) (fst\ k) e^\circ$.

        $(R e)^\circ = \lambda k: (\sim A^\circ \times \sim B^\circ) (snd\ k) e^\circ$.

      \item
        Now, using the intuitionistic logic extended with continuations, we give a typed term proving Peirce's law, which says that $((X \rightarrow Y) \rightarrow X) \rightarrow X$.

        $\lambda f : (X \rightarrow Y)\rightarrow X. letcont\ u: \neg X. throw_X(u, f (\lambda x : X. letcont\ v : \neg Y. throw_Y(u, x)))$

    \end{enumerate}
\end{document}
