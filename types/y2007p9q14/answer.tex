\documentclass[12pt]{article}
\usepackage[margin=1.5cm]{geometry}
\usepackage{parskip}
\usepackage{amsmath}
\usepackage{amssymb}
\usepackage{amsfonts}
\usepackage{enumitem}
\usepackage{graphicx}
\usepackage{stmaryrd}
\graphicspath{ {./images/} }


\begin{document}
\begin{enumerate}[label=(\alph*)]
  \item
  The rules are given below:

  \begin{align*}
    \textsc{var}&\frac{}{\Theta;\Gamma \vdash x : T}(\Gamma(x) = T) \\
    \textsc{fn}&\frac{\Theta;\Gamma, x : T \vdash e : T' \hspace{10pt} \Theta \vdash T type}{\Theta;\Gamma \vdash \lambda x:T. e : T \rightarrow T'}\\
    \textsc{app}&\frac{\Theta;\Gamma \vdash e_1 : T \rightarrow T' \hspace{10pt} \Theta;\Gamma \vdash e_2 : T}{\Theta;\Gamma \vdash e_1 e_2 : T'}\\
    \textsc{$\forall$ fn}&\frac{\Theta,\alpha;\Gamma \vdash e : T}{\Theta;\Gamma \vdash \Lambda \alpha. e : \forall \alpha. T}\\
    \textsc{$\forall$ app}&\frac{\Theta;\Gamma \vdash e : \forall \alpha T \hspace{10pt}\Theta \vdash T' type}{\Theta;\Gamma \vdash e T' : [T' / \alpha]T}\\
  \end{align*}

\item

  A PLC expression $M$ is in beta-normal form if there is no term that it beta-reduces to. I.e. it is not of the form $(\lambda x. e_1) e_2$ or $(\Lambda \alpha. e) T$, and contains no sub-expressions of that form either.

\item
  \begin{enumerate}[label=(\roman*)]
    \item
      We proceed by induction on the structure of $L$ and $N$ to prove that all $L$ and $N$ forms are beta-normal.

      We proceed by cases:

      \begin{itemize}
        \item
          $L' = \lambda x : \tau(L)$

          In this case, we assume that $L$ is in beta-normal form for our induction hypothesis.

          Since $L$ is beta-normal, it contains no beta-redexes, and thus clearly $\lambda x:\tau(L)$ contains no beta-redexes.

        \item
          $L' = \Lambda \alpha(L)$

          Similar to previous case.

        \item
          $L' = N$

          In this case, we assume that $N$ is in beta-normal form for our induction hypothesis, therefore so must $L'$.

        \item
          $N' = x$

          Since $x$ is just a variable, it is not a redex, and thus is in beta-normal form.

        \item
          $N' = N L$

          In this case, we assume that $N$ and $L$ are in beta-normal form for our induction hypothesis.

          Since $N$ and $L$ are beta-normal, in order for $N'$ to not be beta-normal, we would require that $N$ is of the form $\lambda x : \tau(L')$, but this is impossible since the structure of $N'$ requires that it starts with a variable, so $N'$ is beta-normal.

        \item
          $N' = N \tau$

          Similar to previous case.
      \end{itemize}

      Therefore, by induction, all $L$ and $N$ forms are beta-normal.

    \item

      Suppose $\{\} \vdash N : \tau$. By the assumption in the question, $N$ must have no free variables, i.e. they are closed.

      We proceed by induction on the structure of $N$, proving that $N$ terms are not closed.

      By cases:

      \begin{itemize}
        \item
          $N' = x$

          Clearly, this is not a closed expression.

        \item
          $N' = N L$.

          We assume that $N$ is open by our induction hypothesis.

          Clearly, $N L$ will then also be open, regardless of $L$.

        \item
          $N' = N \tau$

          Similar to previous case.
      \end{itemize}

      Therefore, $\{\} \vdash N : \tau$ cannot be true, or else we get a contradiction.

    \item
      Suppose there is some term $L$ such that $\{\} \vdash L : \forall \alpha(\alpha)$

      If this is the case, it must be of the form $\Lambda \alpha (L')$ (or else we would not get this result).

      So, we must have some term $L'$ such that $\{\} \vdash L' : \alpha$

      We proceed by cases on the structure of $L'$:

      \begin{itemize}
        \item
          $L' = \lambda x : \tau (L'')$

          In this case, $L'$ would be of the type $\tau \rightarrow ?$, which is not $\alpha$.

        \item
          $L' = \Lambda \alpha (L'')$

          In this case, $L'$ would be of the type $\forall \alpha(?)$, which is not $\alpha$.

        \item
          $L' = N$

          In this case, $L'$ would be a neutral form, and by (ii) it is not provable that $\{\} \vdash L' : \alpha$.
      \end{itemize}

      Therefore, no such $L'$ term exists, so no such $L$ term can exist.
      
  \end{enumerate}

        
\end{enumerate}
\end{document}
