\documentclass[12pt]{article}
\usepackage[margin=1.5cm]{geometry}
\usepackage{parskip}
\usepackage{amsmath}
\usepackage{amssymb}
\usepackage{amsfonts}
\usepackage{enumitem}
\usepackage{graphicx}
\usepackage{stmaryrd}
\graphicspath{ {./images/} }


\begin{document}
\begin{enumerate}[label=(\alph*)]

    \item
        The Curry-Howard correspondence describes a correspondence between types, programs, etc., and mathematical proofs.

        In particular, we have that types correspond to formulae, and programs correspond to proofs: so having a program of type $T$ is a proof for the corresponding formula for $T$.

        For example, we can consider function types $T_1 \rightarrow T_2$ to be like implication: if we have a proof for $P \implies Q$, and a proof for $P$, we have a proof for $Q$. Similarly, if we have a program of type $T_1 \rightarrow T_2$, and a program of type $T_1$, then we can construct a program of type $T_2$ (through function application).

    \item
        Not relevant

        
    \end{enumerate}
\end{document}
